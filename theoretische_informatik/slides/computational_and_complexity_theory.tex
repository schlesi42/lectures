\documentclass{beamer}
\usetheme{CambridgeUS}
\usepackage{tikz}
\usetikzlibrary{matrix,arrows,fit,positioning, mindmap, trees, automata}

\usepackage[latin1]{inputenc}
\usefonttheme{professionalfonts}
\usepackage{times}
\usepackage{xmpmulti}
\usepackage{animate}
\usepackage{amsmath}
\usepackage{verbatim}
\usepackage{graphicx}
\usepackage{xcolor}
\usepackage{mathrsfs}  
\usepackage{bussproofs}

\usepackage{stmaryrd}
\usepackage{xcolor}

\newcommand{\person}[1]{\textcolor{blue}{#1}}
\newcommand{\highlight}[1]{\textcolor{red}{#1}}
% \newcommand{\example}[1]{\textcolor{blue}{#1}}
\usetikzlibrary{shapes.geometric, positioning}
\graphicspath{ {./images/} }
%\usetheme{Boadilla}
%\usecolortheme{crane}
\title[Computational Theory and Complexity Theory]{Computational Theory and Complexity Theory}
%\subtitle{I Am Curious}
\author[Sebastian Schlesinger]{Prof. Dr.-Ing. Sebastian Schlesinger}
\institute[HWR Berlin]{Berlin School for Economics and Law}
\date{\today}
\begin{document}
 \begin{frame}
\titlepage
\end{frame}



\begin{frame}{Word Problem}
\begin{itemize}
    \item Given a formal language $L$ over an alphabet $\Sigma$ and a word $w \in \Sigma^*$
    \item Decide whether $w \in L$
    \item Example: $L = \{a^n b^n | n \geq 0\}$, $w = aaabbb$
    \item $w \in L$?
\end{itemize}
\end{frame} 

\begin{frame}{Decidability and Semidecidability}
\begin{itemize}
    \item A language $L$ is \textbf{decidable} if there exists a Turing machine that halts on every input and accepts exactly the words in $L$
    \item A language $L$ is \textbf{semidecidable} if there exists a Turing machine that accepts exactly the words in $L$ and may either reject or run forever on words not in $L$
    \item Every decidable language is also semidecidable
    \item Example: The Halting Problem is semidecidable but not decidable
\end{itemize}
\end{frame}

\begin{frame}{Computability}
\begin{itemize}
    \item A function $f: \Sigma^* \rightarrow \Sigma^*$ is \textbf{computable} if there exists a Turing machine that, given any input $w \in \Sigma^*$, halts and outputs $f(w)$
    \item Example: The function that maps a binary number to its successor is computable
    \item Not all functions are computable (e.g., the Halting Problem)
\end{itemize}
\end{frame}

\begin{frame}{Completeness and Decidability}
 The propositional logic is decidable. However, first-order logic is not decidable, but semi-decidable.
 However, first-order logic is complete, i.e., every valid formula can be proven.
 Hence, although there must be a proof for every valid formula, there is no algorithm that can find it for every formula in finite time.
\end{frame}

\begin{frame}{Decidability of the Chomsky Hierarchy}
\begin{itemize}
    \item Type 3 (Regular Languages): Decidable
    \item Type 2 (Context-Free Languages): Decidable
    \item Type 1 (Context-Sensitive Languages): Decidable
    \item Type 0 (Recursively Enumerable Languages): Semi-decidable
\end{itemize}
\end{frame}

\begin{frame}{Recursively Enumerable vs. Recursive Languages}
\begin{itemize}
    \item \textbf{Recursive Languages}: Languages for which there exists a Turing machine that halts on all inputs and decides membership.
    \item \textbf{Recursively Enumerable Languages}: Languages for which there exists a Turing machine that will enumerate all valid strings but may not halt on invalid strings.
\end{itemize}
\end{frame}

\begin{frame}{Primitive Recursive Functions}
\begin{itemize}
    \item A class of functions that can be defined using basic functions (zero, successor, projection) and closed under composition and primitive recursion.
    \item All primitive recursive functions are total and computable.
    \item Example: Addition, multiplication, and factorial are primitive recursive functions.
\end{itemize}
\end{frame}

\begin{frame}{Mu-Recursive Functions}
\begin{itemize}
    \item An extension of primitive recursive functions that includes the minimization operator.
    \item Mu-recursive functions can express a wider range of computations, including some that are not primitive recursive.
    \item All mu-recursive functions are computable, but not all are total.
\end{itemize}
\end{frame}

\begin{frame}{Church-Turing Thesis}
\begin{itemize}
    \item The Church-Turing Thesis posits that any function that can be computed by an algorithm can be computed by a Turing machine.
    \item This thesis provides a foundation for the field of computability theory.
    \item It implies that Turing machines are a powerful model of computation, equivalent to other models (e.g., lambda calculus, recursive functions).
\end{itemize}
\end{frame}

\begin{frame}{Computability vs. Complexity}
\begin{itemize}
    \item \textbf{Computability} is concerned with what can be computed (i.e., the existence of an algorithm).
    \item \textbf{Complexity} is concerned with how efficiently a problem can be solved (i.e., the resources required).

\end{itemize}
\end{frame}

\begin{frame}{Complexity Classes}
\begin{itemize}
    \item \textbf{P}: Class of decision problems solvable by a deterministic Turing machine in polynomial time.
    \item \textbf{NP}: Class of decision problems solvable by a nondeterministic Turing machine in polynomial time. Equivalently, problems for which a given solution can be verified in polynomial time.
    \item \textbf{PSPACE}: Class of decision problems solvable by a Turing machine using polynomial space.
    \item \textbf{EXPTIME}: Class of decision problems solvable by a deterministic Turing machine in exponential time.
\end{itemize}
\end{frame} 
\begin{frame}{P vs NP Problem}
\begin{itemize}
    \item One of the most important open problems in computer science.
    \item Question: Is P equal to NP?
    \item If P = NP, then every problem for which a solution can be verified quickly can also be solved quickly.
    \item Most experts believe that P $\neq$ NP, but it remains unproven.
\end{itemize}
\end{frame} 
\begin{frame}{NP-Hard and NP-Complete}
\begin{itemize}
    \item A problem is \textbf{NP-hard} if every problem in NP can be reduced to it in polynomial time.
    \item A problem is \textbf{NP-complete} if it is both in NP and NP-hard.
    \item If any NP-complete problem can be solved in polynomial time, then P = NP.
    \item Examples of NP-complete problems: Traveling Salesman Problem, Boolean Satisfiability Problem (SAT), Knapsack Problem
\end{itemize}
\end{frame}
\end{document}