\documentclass{beamer}
\usetheme{CambridgeUS}
\usepackage{tikz}
\usepackage[german]{babel}
\usetikzlibrary{matrix,arrows,fit,positioning, mindmap, trees}

\usepackage[latin1]{inputenc}
\usefonttheme{professionalfonts}
\usepackage{times}

\usepackage{xmpmulti}
\usepackage{animate}
\usepackage{amsmath}
\usepackage{verbatim}
\usepackage{graphicx}
\usepackage{xcolor}
\usepackage{mathrsfs}  
\usepackage{bussproofs}
\usepackage{tikz}
\usepackage{stmaryrd}
\usepackage{xcolor}
\newcommand{\person}[1]{\textcolor{blue}{#1}}
\newcommand{\highlight}[1]{\textcolor{red}{#1}}
% \newcommand{\example}[1]{\textcolor{blue}{#1}}
\usetikzlibrary{shapes.geometric, positioning}
\graphicspath{ {./images/} }
%\usetheme{Boadilla}
%\usecolortheme{crane}
\title[Diskrete Mathematik]{Wiederholung Diskrete Mathematik (aus Vorlesung Semester 1, Theoretische Grundlagen der Informatik)}

\author[Sebastian Schlesinger]{Prof. Dr.-Ing. Sebastian Schlesinger}
\institute[HWR Berlin]{Berlin School for Economics and Law}
\date{\today}
\begin{document}
 \begin{frame}
\titlepage
\end{frame}



\begin{frame}
  \frametitle{Outline}
  \tableofcontents
\end{frame}
\section{Aussagenlogik}
\begin{frame}{Agenda}
   
    
  \tableofcontents[currentsection]
  \end{frame}
\begin{frame}
\frametitle{Aussagen}

Unter einer \textbf{Aussage} versteht man einen sprachlichen Ausdruck, dem man eindeutig einen der beiden Wahrheitswerte w (\glqq wahr\grqq) bzw. f (\glqq falsch\grqq) zuordnen kann. 

Aussagen werden mit Gro\ss buchstaben bezeichnet, \[A:Beschreibung\]
und k"onnen mit logischen Operationen verkn"upft werden. Grundlegende mathematische Aussagen, die nicht aus anderen Aussagen abgeleitet werden k"onnen, nennt man \textbf{Axiome}.
\end{frame}



\begin{frame}
  \frametitle{Logische Operationen}
  Logische Aussagen k"onnen durch die in der folgenden Tabelle angegebenen Operationen verkn"upft werden.
  
  \begin{tabular}[h]{c|c|c}
   
    Bezeichnung & Formelzeichen & Semantik (Aussage wahr gdw) \\
    \hline
    Negation & $\neg A$ &  A falsch ist \\
    Konjunktion & $A\wedge B$ &  A und B wahr sind \\
    Disjunktion & $A\vee B$ &  A oder B wahr ist \\
    Implikation & $A\Rightarrow B$ &  A falsch oder B wahr \\
    "Aquivalenz & $A\Leftrightarrow B$ & A und B "aquivalent 


    \end{tabular}
\end{frame}


\begin{frame}
  \frametitle{Wahrheitstabelle}
  In der folgenden Tabelle sind die Wahrheitswerte der vorgestellten Verkn"upfungen angegeben. Dabei steht w f"ur wahr und f f"ur falsch.
  \begin{tabular}[h]{c|c|c|c|c|c|c}
   
    $A$ & $B$ & $\neg A$ & $A\wedge B$ & $A\vee B$ & $A\Rightarrow B$ & $A\Leftrightarrow B$ \\
    \hline
    w & w & f & w & w & w & w \\
    w & f & f & f & w & f & f \\
    f & w & w & f & w & w & f \\
    f & f & w & f & f & w & w
    
  \end{tabular}
  
  \vspace{5mm}
  \textbf{Hinweis:} Statt $w$ f"ur \textit{wahr} und $f$ f"ur \textit{falsch} werden auch die Symbole $\top$ und $\bot$ verwendet.
  
\end{frame}

\begin{frame}
  \frametitle{Gesetze f"ur logische Operationen}
  F"ur logische Operationen gelten die folgenden Identit"aten. 
  \begin{itemize}
    \item Assoziativgesetze: \[(A\wedge B)\wedge C = A\wedge(B\wedge C)\] \[(A\vee B)\vee C = A\vee(B\vee C)\]
    \item Kommutativgesetze: \[A\wedge B = B\wedge A\]\[A\vee B = B\vee A\]
    \item Distributivgesetze: \[A \wedge (B\vee C) = (A\wedge B)\vee (A\wedge C)\] \[A\vee (B\wedge C) = (A\vee B)\wedge (A\vee C)\]
  \end{itemize}
   
\end{frame}


\begin{frame}
  \frametitle{Gesetze f"ur logische Operationen}
  F"ur logische Operationen gelten die folgenden Identit"aten. 
  \begin{itemize}
    \item De Morgansche Regeln: \[\neg(A\wedge B) = (\neg A)\vee(\neg B)\]\[\neg(A\vee B) = (\neg A)\wedge(\neg B)\]
    \item Idempotenz: \[\neg(\neg A) = A\]\[A\vee A = A\]\[A\wedge A = A\]
    \item K"urzungsregeln: \[A\vee\bot=A\]\[A\wedge\top=A\]
  \end{itemize}
   
\end{frame}

\section{Pr"adikatenlogik}
\begin{frame}{Agenda}
   
    
  \tableofcontents[currentsection]
  \end{frame}
\begin{frame}{Quantoren}
 
  \begin{definition}[Quantoren]
    Wir definieren folgende Notation:
    \begin{itemize}
      \item Mit $\forall x:A(x)$ definieren wir, dass eine Aussage $A$, die eine Variable $x$ beinhaltet, f"ur alle Werte von $x$ gelten soll ("ublicherweise wird dann auch eine Grundmenge, aus denen die $x$ stammen sollen, angegeben).
      \item Mit $\exists x:A(x)$ dr"ucken wir aus, dass es ein $x$ geben soll, so dass $A(x)$ gilt.
    \end{itemize}
  \end{definition}

 
\end{frame}

\begin{frame}{Beispiele pr\"adikatenlogischer Formeln}
  Sei $f:D\to \mathbb{R}$ mit $D\subseteq\mathbb{R}$ eine Funktion. Dann ist $f$ in $x_0\in D$ \textbf{stetig}, wenn \[\forall\varepsilon>0\exists\delta>0\forall x\in D:|x-x_0|<\delta\Rightarrow |f(x)-f(x_0)|<\varepsilon\]

  Um auszudr"ucken, dass genau ein Element in einer Menge existiert, kann man das wie folgt formulieren: \[\exists! x\in M: A(x):\Leftrightarrow \exists x\in M: A(x) \wedge \forall y\in M: A(y)\Rightarrow y=x\]

\end{frame}

\section{Diskrete Mathematik}
\subsection{Mengenlehre}
\begin{frame}{Agenda}
   
    
  \tableofcontents[currentsection]
  \end{frame}
\begin{frame}
  \frametitle{Mengendefinition}
  \begin{definition}[Naive Mengendefinition]
    Eine Menge ist die Zusammenfassung von bestimmten unterschiedlichen
    Objekten (die Elemente der Menge) zu einem neuen Ganzen.
    Wir schreiben $x\in M$, falls das Objekt $x$ zur Menge $M$ geh"ort.
    Wir schreiben $x\notin M$, falls das Objekt $x$ nicht zur Menge $M$ geh"ort.
    Falls $x\in M$ und $y\in M$ gilt, schreiben wir auch $x, y \in M$.
    Eine Menge, welche nur aus endlich vielen Objekten besteht (eine endliche
    Menge), kann durch explizite Auflistung dieser Elemente spezifiziert
    werden.
  \end{definition}
    Beispiel: $M=\{2,3,5,7\}$.

    Hierbei spielt die Reihenfolge der Auflistung keine Rolle:
    \[\{2,3,5,7\}=\{7,5,3,2\}\]
    Auch Mehrfachauflistungen spielen keine Rolle:
    \[\{2,3,5,7\}=\{2,2,2,3,3,5,7\}\]
  \end{frame}

\begin{frame}{Mengennotation}
Mengen k"onnen definiert werden durch
\begin{itemize}
  \item Aufz"ahlung der Elemente
  \item Formulierung von Bedingungen in der Form $M:=\{x|p(x)\}$, wobei $p$ ein \textit{Pr"adikat}, also eine Aussage ist, die $x$ enth"alt, so dass man jeweils bei Einsetzen von $x$ entscheiden kann, ob sie wahr (und damit das Element zur Menge geh"ort) oder falsch ist (und damit das Element $x$ nicht zu $M$ geh"ort).
\end{itemize}
  Wir schreiben zur Abk"urzung auch $M:=\{x\in N|p(x)\}$ statt $M:=\{x|x\in N\wedge p(x)\}$.
\end{frame}
\begin{frame}
  \frametitle{Besondere Mengen}
  Eine besonders wichtige Menge ist die leere Menge $\emptyset = \{\}$, die keinerlei
Elemente enth"alt.

In der Mathematik hat man es h"aufig auch mit unendlichen Mengen zu
tun (Mengen, die aus unendlich vielen Objekten bestehen).
Solche Mengen k"onnen durch Angabe einer Eigenschaft, welche die
Elemente der Menge auszeichnet, spezifiziert werden.

Beispiele:
\begin{itemize}
  \item $\mathbb{N}=\{0,1,2,3,\dots\}$
  \item $\mathbb{Z}=\{\dots,-2,-1,0,1,2,\dots\}$
  \item $\mathbb{Q}=\{\frac{p}{q}|p,q\in\mathbb{Z}, q\neq 0\}$
  
\end{itemize}
\end{frame}

\begin{frame}{Warum naive Mengenlehre?}
  Die \glqq Definition\grqq der Menge ist anf"allig f"ur Widerspr"uche, z.B. die \textbf{Russelsche Antinomie}:

  Man bilde die Menge aller Mengen, die sich nicht selbst als Element enthalten, in Formeln:
  \[M:=\{N|N\notin N\}\]
  Frage: Gilt $M\in M$? Das f"uhrt auf einen Widerspruch.
  
  Daher hat man die Mengenlehre mit dem \textbf{Zermelo-Fraenkelschen Axiomensystem} auf ein solides Fundament gehoben. Mehr dazu im optionalen Inhalt (ist zu kompliziert f"ur eine erste Einf"uhrung).
\end{frame}

\begin{frame}{Teilmengen}
  \begin{definition}[Teilmenge]
    Seien $A$ und $B$ Mengen. $A\subseteq B$ bedeutet $A$ ist \textit{Teilmenge} von $B$, genau dann, wenn 
    \[\forall x:x\in A\Rightarrow x\in B\] oder "aquivalent dazu \[\forall x\in A:x\in B\]
  \end{definition}
  
  \begin{lemma}[Gleichheit von Mengen]
    Seien $A$ und $B$ Mengen. Es ist $A=B$ genau dann, wenn $\forall x:x\in A\Leftrightarrow x\in B$, was wiederum "aquivalent ist zu
    \[A\subseteq B\wedge B\subseteq A\]  
  \end{lemma}
  Um also die Gleichheit von zwei Mengen zu zeigen beweist man "ublicherweise erst $A\subseteq B$ und dann $B\subseteq A$.
\end{frame}

\begin{frame}{Mengenoperationen}
  \begin{definition}[Mengenoperationen]
    Seien $A$ und $B$ Mengen. Wir definieren:
    \begin{itemize}
      \item $A\cap B:=\{x|x\in A\wedge x\in B\}$, den \textit{Schnitt} von $A$ und $B$
      \item $A\cup B:=\{x|x\in A\vee x\in B\}$, die \textit{Vereinigung} von $A$ und $B$
      \item $A\backslash B:=\{x\in A|x\notin B\}$, die \textit{Differenz} von $A$ und $B$
    \end{itemize}
  \end{definition}
\end{frame}

\begin{frame}{Venn-Diagramme}
  Die Operationen lassen sich in \textbf{Venn-Diagrammen} visualisieren.
  \begin{tikzpicture}
 
    % Set A
    \node [draw,
        circle,
        minimum size =3cm,
        label={135:$A$}] (A) at (0,0){};
     
    % Set B
    \node [draw,
        circle,
        minimum size =3cm,
        label={45:$B$}] (B) at (1.8,0){};
     
    % Set intersection label
    \node at (0.9,0) {$A\cap B$};
     
    \end{tikzpicture}
    \begin{tikzpicture}
 
      % Set A
      \node [circle,
          fill=orange,
          minimum size =3cm,
          label={135:$A$}] (A) at (0,0){};
       
      % Set B
      \node [circle,
          fill=orange,
          minimum size =3cm,
          label={45:$B$}] (B) at (1.8,0){};
       
      % Circles outline
      \draw[white,thick] (0,0) circle(1.5cm);
      \draw[white,thick] (1.8,0) circle(1.5cm);
       
      % Union text label
      \node[orange!90!black] at (0.9,1.8) {$A\cup B$};
       
      \end{tikzpicture}
    
      \begin{tikzpicture}[thick,
        set/.style = { circle, minimum size = 3cm}]
     
    % Set A
    \node[set,fill=orange,label={135:$A$}] (A) at (0,0) {};
     
    % Set B
    \node[set,fill=white,label={45:$B$}] (B) at (0:2) {};
     
    % Circles outline
    \draw (0,0) circle(1.5cm);
    \draw (2,0) circle(1.5cm);
     
    % Difference text label
    \node[left,white] at (A.center){$A-B$};
     
    \end{tikzpicture}
\end{frame}

\begin{frame}{Weitere Mengen und Eigenschaften}
  \begin{definition}[Potenzmenge]
    Mit \[\mathscr{P}(A):=2^A:=\{B|B\subseteq A\}\] bezeichnen wir die \textbf{Potenzmenge}, die Menge aller Teilmengen von $A$.

  \end{definition}
  \begin{definition}[Disjunktheit]
    Zwei Mengen $A$ und $B$ sind \textit{disjunkt}, falls $A\cap B=\emptyset$ gilt.
  \end{definition}
\end{frame}

\begin{frame}{Beispiele f"ur Mengenaussagen}
Es gilt:
\begin{itemize}
  \item $\forall A:\emptyset\subseteq A$
  \item $\forall A:A\subseteq A$
  \item $\mathbb{N}\subseteq\mathbb{Z}\subseteq\mathbb{Q}\subseteq\mathbb{R}$
  \item $\{1,2,3\}\cap\{4,5,6\}=\emptyset$
  \item $\mathscr{P}(\{1,2\})=\{\emptyset,\{1\},\{2\},\{1.2\}\}$
  \item $\mathscr{P}(\emptyset)=\{\emptyset\}$
  \item $\forall A: A\cap\emptyset=\emptyset$
  \item $\forall A:A\cup\emptyset=A$
  \item $\forall A,B,C: A\cap(B\cup C)=(A\cap B)\cup (B\cap C)$
  \item $\forall A,B,C: A\cup (B\cap C)=(A\cup B)\cap (A\cup C)$
  \item $\forall A,B,C: A\backslash(B\cup C)=(A\backslash B)\cap (A\backslash C)$
  \item $\forall A,B,C: A\backslash(B\cap C)=(A\backslash B)\cap (A\backslash C)$
\end{itemize}

\end{frame}
\subsection{Relationen}
\begin{frame}{Agenda}
   
    
  \tableofcontents[currentsection]
  \end{frame}
\begin{frame}{Kartesisches Produkt}
  Zun"achst f"uhren wir Paare ein. Diese werden, im Gegensatz zu Mengen mit runden Klammern notiert, z.B.
  $(1,2)$ das Paar, dessen erste Komponente die Zahl $1$ und zweite Komponente $2$ ist. Hier kommt es auf die Reihenfolge an, z.B. $(1,2)\neq(2,1)$, aber $\{1,2\}=\{2,1\}$, weil 
  es bei Mengen nur auf die Elemente, nicht auf deren Reihenfolge ankommt.
  
  Es gilt insbesondere \[(a,b)=(c,d)\Leftrightarrow a=c\wedge b=d\]
  \begin{definition}[Kartesisches Produkt]
    Seien $M,N$ Mengen. Dann ist \[M\times N:=\{(x,y)|x\in M,y|in N\}\] das \textbf{kartesische Produkt} von $M$ und $N$.
  \end{definition}
  
\end{frame}

\begin{frame}{Relationen}
 Relationen sind intuitiv Strukturen, um Beziehungen zwischen Objekten auszudr"ucken.
 \begin{definition}[Relationen]
  Seien $M$ und $N$ Mengen. Eine Teilmenge $R$ des kartesischen Produktes $M\times N$, also 
  \[R\subseteq M\times N\] nennen wir \textbf{Relation} von $M$ auf $N$.
 \end{definition}
\end{frame}

\begin{frame}{Beispiele von Relationen}  
Ist $M$ die Menge der Menschen, dann ist $R\subseteq M\times M$ die Eltern-Kind-Beziehung ein Beispiel f"ur eine Relation.
Also $(x,y)\in R\Leftrightarrow x$ ist Elternteil von $y$. 

Ein anderes Beispiel ist die $\leq$ Relation auf $\mathbb{N}$, also $(x,y)\in R\subseteq \mathbb{Z}\times\mathbb{Z}\Leftrightarrow x\leq y$.

Generell kann man eine Relation zwischen zwei Elementen auf zwei Arten notieren: auf die herk"ommliche Weise, also $(x,y)\in R$ oder mittels \textbf{Infixschreibweise} $xRy$.
\end{frame}
  

\begin{frame}{Darstellung von Relationen}
  Eine M"oglichkeit ist die Angabe als Menge, z.B. 
  \[R=\{(A,B), (A,A), (B,C), (C,B)\}\]

  Eine Alternative ist ein Graph.

  \begin{tikzpicture}{center}
    \node[shape=circle,draw=black] (A) at (2,2) {A};
    \node[shape=circle,draw=black] (B) at (0,0) {B};
    \node[shape=circle,draw=black] (C) at (3,0) {C};
    

    \path [->] (A) edge (B);
    \path [->] (B) edge  (C);
    \path [->] (C) edge  (B);
    \path[<->] (A) edge [loop left]  (A);
    
    
\end{tikzpicture}



\end{frame}
\begin{frame}{Darstellung von Relationen}
  Eine andere ist die einer Matrix.
  \[
  \begin{pmatrix}
    $1$ & $1$ & $0$ \\
    $0$ & $0$ & $1$ \\
    $0$ & $1$ & $0$
  \end{pmatrix}
  \]
  Hier kann man sich die Zeilen und Spalten der Matrix annotiert mit den Elementen der Grundmengen denken.
  Etwas formaler: Eine \textbf{Adjazenzmatrix} einer Relation $R\subseteq M\times N$ mit $M=\{x_1,...,x_m\}$ und $N=\{y_1,...,y_n\}$, also $m$ Elementen in $M$ und $n$ Elementen in $N$, die man sich geordnet vorstellen kann ("uber die Indizierung) ist eine Matrix $(a_{ij})$ mit $1\leq i\leq m, 1\leq j\leq n$ und f"ur die Eintr"age $a_{ij}$ in Zeile $i$ und Spalte $j$ gilt:
  \[a_{ij}=1\Leftrightarrow (x_i,y_j)\in R, a_{ij}=0\Leftrightarrow (x_i,y_j)\notin R \]
\end{frame}

\begin{frame}{Eigenschaften von Relationen}
Relationen sind in h"ochster Allgemeinheit definiert. Nun kann man bestimmte Eigenschaften untersuchen.
\begin{definition}
Sei $R\subseteq M\times M$ eine Relation "uber einer Menge $M$ (Man beachte, dass die Elemente nur aus einer Menge $M$ stammen). Dann ist $R$
\begin{enumerate}
\item \textbf{reflexiv}, wenn $\forall x\in M: (x,x)\in R$
\item \textbf{irreflexiv}, wenn $\forall x\in M: (x,x)\notin R$
\item \textbf{symmetrisch}, wenn $\forall x,y\in M: (x,y)\in R\Rightarrow (y,x)\in R$
\item \textbf{antisymmetrisch}, wenn $\forall x,y\in M: (x,y)\in R\wedge (y,x)\in R\Rightarrow x=y$
\item \textbf{asymmetrisch}, wenn $\forall x,y\in M: (x,y)\in R\Rightarrow (y,x)\notin R$
\item \textbf{transitiv}, wenn $\forall x,y,z\in M: (x,y)\in R\wedge (y,z)\in R\Rightarrow (x,z)\in R$
\end{enumerate}
\end{definition}
\end{frame}

\begin{frame}{Komposition von Relationen}
\begin{definition}[Komposition von Relationen]
  Seien $R\subseteq M\times N, S\subseteq N\times P$. Wir definieren 
  \[R\circ S=\{(x,z)\in M\times P|\exists y\in N:(x,y)\in R\wedge (y,z)\in S\}\]
  als \textbf{Komposition} von $R$ und $S$.

\end{definition}
Die Komposition ist praktisch die \glqq Hintereinanderschaltung\grqq der Relationen $R$ und $S$.
\end{frame}


\begin{frame}{Algorithmische Berechnung der Komposition (Matrixmultiplikation)}
  
  \begin{example}
    Seien $R$ und $S$ zwei Relationen auf den Mengen $M$ und $N$ mit den Adjazenzmatrizen:
    \[
    R = \begin{pmatrix}
      1 & 0 & 1 \\
      0 & 1 & 0 \\
      1 & 0 & 0
    \end{pmatrix}
    \quad \text{und} \quad
    S = \begin{pmatrix}
      0 & 1 & 0 \\
      1 & 0 & 1 \\
      0 & 1 & 0
    \end{pmatrix}
    \]
    Sei $R=(\alpha_{i,j})_{1\leq i,j\leq n}$, $S=(\beta_{i,j})_{1\leq i,j\leq n}$
    Um das $(i,j)$-te Element von $R\circ S$, $\gamma_{i,j}$ zu berechnen, definieren wir $\gamma_{i,j}=\bigvee_{\nu=1}^n \alpha_{i,\nu}\wedge\beta_{\nu,j}$.
    Das bedeutet, dass wir das Element der Zielmatrix in Zeile $i$ und Spalte $j$ so erhalten, indem wir die $i-$te Zeile von $R$ und die $j-$te Spalte von $S$ durchgehen und suchen, ob $1$ auf $1$ trifft. Finden wir ein Paar, wird das Zielelement $1$, sonst $0$.
    
  \end{example}
\end{frame}


\begin{frame}{Ordnungen und "Aquivalenzrelationen}
  \begin{definition}[Ordnung]
    Eine Relation $R\subseteq M\times M$ hei"st \textbf{Ordnung}, wenn sie reflexiv, antisymmetrisch und transitiv ist.
  \end{definition}
  \begin{definition}["Aquivalenzrelation]
    Eine Relation $R\subseteq M\times M$ hei"st \textbf{"Aquivalenzrelation}, wenn sie reflexiv, symmetrisch und transitiv ist.
  \end{definition}
\end{frame}


  \begin{frame}{Hasse-Diagramme}
    Hasse-Diagramme sind eine spezielle Art von Graphen, die Ordnungen visualisieren. Sie sind besonders n"utzlich, um die Struktur von Ordnungen zu verdeutlichen.

    \begin{tikzpicture}
      \node (b) at (0,0) {b};
      \node (d) at (1,1) {d};
      \node (c) at (2,0) {c};
      \node (a) at (1,-1) {a};

      \draw (a) -- (b);
      \draw (a) -- (c);
      \draw (b) -- (d);
      \draw (c) -- (d);
    \end{tikzpicture}

    In dem Beispiel ist die Ordnung $\{(a,a),(a,b),(a,c),(a,d),(b,b),(b,d),(c,c),(c,d),(d,d)\}$ dargestellt.

    Die Idee ist, die reflexiven und transitiven Beziehungen nicht darzustellen. Sie gelten implizit. Und gilt $aRb$ in der Ordnung $R$, dann wird $a$ unterhalb von $b$ angeordnet und beide werden mit einer Linie verbunden.
    
  \end{frame}
  \begin{frame}{Partitionen}
    \begin{definition}[Partition]
      Eine Partition einer Menge $M$ ist eine Menge von nicht-leeren Teilmengen $\{A_i\}_{i\in I}$, so dass
      \begin{itemize}
        \item $\bigcup_{i\in I} A_i = M$ (die Vereinigung aller Teilmengen ergibt die Menge $M$)
        \item $A_i \cap A_j = \emptyset$ f"ur alle $i \neq j$ (die Teilmengen sind paarweise disjunkt)
      \end{itemize}
    \end{definition}
    \begin{example}
      Sei $M = \{1, 2, 3, 4\}$. Eine m"ogliche Partition von $M$ ist $\{\{1, 2\}, \{3, 4\}\}$.
    \end{example}
  \end{frame}
\begin{frame}{Zusammenhang von Partitionen und "Aquivalenzrelationen}
  \begin{itemize}
    \item Jede "Aquivalenzrelation auf einer Menge $M$ definiert eine Partition von $M$.
    \item Umgekehrt definiert jede Partition einer Menge $M$ eine "Aquivalenzrelation auf $M$.
  \end{itemize}
  \begin{example}
    Sei $M = \{1, 2, 3, 4\}$ und die Partition $\{\{1, 2\}, \{3, 4\}\}$. Die entsprechende "Aquivalenzrelation ist:
    \[
    R = \{(1, 1), (1, 2), (2, 1), (2, 2), (3, 3), (3, 4), (4, 3), (4, 4)\}
    \]
  \end{example}
\end{frame}
\begin{frame}{Zusammenhang von Partitionen und "Aquivalenzrelationen}

  \begin{definition}[Menge der "Aquivalenzklassen]
    Sei $R$ eine "Aquivalenzrelation auf einer Menge $M$. Die Menge der "Aquivalenzklassen von $M$ bzgl. $R$ (auch bezeichnet als Quotienten- oder Faktormenge) ist definiert als:
    \[M/R := \{[x]_R \mid x \in M\}\]
    wobei $[x]_R := \{y \in M \mid (x, y) \in R\}$ die "Aquivalenzklasse von $x$ bzgl. $R$ ist.
  \end{definition}
  Es gilt, dass zu jeder "Aquivalenzrelation $R\subseteq M\times M$ die Menge der "Aquivalenzklassen $M/R$ eine Partition bildet. Umgekehrt gibt es zu jeder Partition $P\subseteq\mathscr{P}(M)$ eine passende "Aquivalenzrelation $R\subseteq M\times M$, so dass also $M/R=P$.
\end{frame}

\begin{frame}{Reflexiv-transitive H"ulle}
Zu jeder Relation $R\subseteq M\times M$ l"asst sich die reflexiv-transitive H"ulle $R^*$ bilden. Das ist die kleinste (im Sinne $\subseteq$) $R$ enthaltende Relation, die reflexiv und transitiv ist.

Es gilt $R^*=\bigcup_{n=0}^\infty R^n$ mit $R^0=Id$, $R^{n+1}=R^n\circ R$. Dabei ist $Id=\{(x,x)|x\in M\}$ die Relation, die jedes Element und nur diese mit sich selbst in Beziehung setzen. 

Bei endlichen Mengen kann man $R^*$ berechnen, indem man so lange $R^n$ hinzuf"ugt bis $\bigcup{i=0}^n R^i$ station"ar wird.
\end{frame}
\subsection{Funktionen}

\begin{frame}{Agenda}
   
    
  \tableofcontents[currentsection]
  \end{frame}
\begin{frame}{Funktionen}
  Funktionen sind spezielle Relationen, die jedem Element im Urbildraum (der ersten Menge im kartesischen Produkt) genau ein ELement im Bildraum zuordnen.
  \begin{definition}[Funktionen]
    Eine Relation $f\subseteq M\times N$ hei"st \textbf{Funktion}, wenn \[\forall x\in M \exists y\in N: (x,y)\in f\wedge \forall y'\in N:(x,y')\in f\Rightarrow y=y'\]
    Da somit das mit einem Element $x$ in Relation stehende Element $y$ eindeutig bestimmt ist, schreiben wir auch $y=f(x)$ und bezeichnen $y$ als \textbf{das Bild} von $x$.
  \end{definition}
  Wir notieren eine Relation $f\subseteq M\times N$, die eine Funktion ist auch mit $f:M\to N, x\mapsto f(x)$.
\end{frame}


  \begin{frame}{Ordnungen und "Aquivalenzrelationen}
    \begin{definition}[Ordnungen]
      Eine Relation $R\subseteq M\times M$ hei"st \textbf{Ordnung}, wenn sie reflexiv, antisymmetrisch und transitiv ist.
    \end{definition}
    \begin{definition}["Aquivalenzrelationen]
      Eine Relation $R\subseteq M\times M$ hei"st \textbf{"Aquivalenzrelation}, wenn sie reflexiv, symmetrisch und transitiv ist.
    \end{definition}
  \end{frame}
\begin{frame}{Eigenschaften von Funktionen}
\begin{definition}[Eigenschaften von Funktionen]
  Eine Funktion $f:M\to N$ hei"st 
  \begin{itemize}
    \item \textbf{injektiv}, wenn $\forall x_1,x_2\in M:f(x_1)=f(x_2)\Rightarrow x_1=x_2$.
    \item \textbf{surjektiv}, wenn $\forall y\in N\exists x\in M:y=f(x)$.
    \item \textbf{bijektiv}, wenn $f$ injektiv und surjektiv ist.
  \end{itemize}
\end{definition}
Intuitiv bedeutet \textit{injektiv}, dass ein Funktionswert h"ochstens einen Ursprungswert hat, oder anders: Jeder Wert im Zielraum wird h"ochstens einmal von $f$ \glqq getroffen\grqq. \textit{Surjektiv} bedeutet, dass jeder Wert von $f$ \glqq getroffen\grqq wird.
\end{frame}
\begin{frame}{Urbilder von Funktionen}
  \begin{definition}[Urbild]
    Sei $f:M\to N$ eine Funktion und $B\subseteq N$. Dann hei"st \[f^{-1}(B):=\{x\in M|f(x)\in B\}\] das \textbf{Urbild} von $B$ unter $f$.
  \end{definition}
  Beachte, dass $f^{-1}$ hier nicht die Umkehrfunktion bezeichnet, sondern nur das Urbild. Das Urbild ist immer wohldefiniert, auch wenn $f$ nicht bijektiv ist.
  \end{frame}

\begin{frame}{Bemerkungen zu Funktionen und Bild der Funktion}
Beachte, dass $f:X\to Y$ immer eine \textit{totale} Funktion ist, also jedes Element in $X$ ein Bild in $Y$ hat. Hat nicht jedes Element von $X$ ein Bild, so muss man die Definition entsprechend einschränken. Manche Autoren lassen aber auch partielle Funktionen zu, also solche, die nicht jedes Element in $X$ abbilden. Wir erwähnen das, falls notwendig.

\begin{definition}{Bild einer Funktion}
  Sei $f:X\to Y$ eine Funktion. Dann hei"st \[Im(f):=f(X)=\{y\in Y|\exists x\in X:y=f(x)\}\] das \textbf{Bild} von $X$ unter $f$.
\end{definition}
\end{frame}
\begin{frame}{Kanonische Faktorisierung einer Abbildung}
  Sei $f:X\to Y$ eine Funktion. Wir definieren die Relation $R\subseteq X\times X$ durch \[(x_1,x_2)\in R\Leftrightarrow f(x_1)=f(x_2).\] Dann ist $R$ eine \"Aquivalenzrelation, und wir k"onnen die kanonische Faktorisierung von $f$ betrachten. Diese ist gegeben durch die Abbildung \[\bar{f}:X/R\to Im(f), [x]\mapsto f(x).\] Dabei ist $[x]$ die \"Aquivalenzklasse von $x$ unter $R$.
  \textit{Kanonische Faktorisierung} bedeutet, dass f\"ur die Projektion \[\pi:X\to X/R, x\mapsto[x]\] gilt: \[f=\bar{f}\circ\pi.\]
\end{frame}

\begin{frame}{Beispiel f"ur kanonische Faktorisierung}
Sei $X = \{1,2,3,4,5,6,7,8\}, Y=\{a,b,c\}$, und $f:X\to Y$ definiert durch: $f(1) = a$, $f(2) = a$, $f(3) = c$, $f(4) = b$, $f(5) = a$, $f(6) = b$, $f(7) = c$, $f(8) = a$. 
  \begin{enumerate}[(i)]
    \item Was ist $f^{-1}(\{a\})$, $f^{-1}(\{b\})$, $f^{-1}(\{c\})$? Das ergibt eine Partition von $X$.
    \item Definiere $x_1\sim x_2$ falls $f(x_1) = f(x_2)$. Was sind die \"Aquivalenzklassen? 
    \item Beschreibe $\pi : X\to X/\sim$. 
    \item Nun widmen wir uns $\bar{f} : X/\sim\to Y$, definiert durch $\bar{f}([x]) = f(x)$. Macht euch klar, dass es eine Bijektion ist. 
   \item Macht Euch klar, dass $f = \bar{f}\circ\pi$ gilt
  \end{enumerate}
\end{frame}

  \end{document}