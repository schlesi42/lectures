%!TEX TS-program = pdflatex
%!TEX TS-options = -shell-escape
% % % % %   Die folgenden Zeilen müssen ihre Zeilennummern 4 und 5 behalten !!!    % % % % %
\newcommand{\printpraesenzlsg}{false}
\newcommand{\printloesungen}{false}
\newcommand{\printbewertungen}{false}
% % % % %   \newcommand{\printloesungen}{false}                                    % % % % %
\newcommand{\blattnummer}{1}
%\newcommand{\abgabetermin}{\textcolor{red}{bis 11.04.2022, 09:00 Uhr}}
\input{include/config.tex}

% Änderungen 2020: Hinweise auf Digitallehre angepasst; Rechnergeschichte entfernt; Blatt 1 und 2 zusammengefasst.
\begin{document}
\iforiginal{\thispagestyle{scrplain}
\vspace*{-3cm}
\begin{minipage}[t][1.1cm][c]{4.5cm}
  \includegraphics[width=4cm]{include/hwr-logo.png}
\end{minipage}
\hfill
\begin{minipage}[t][1.5cm][c]{8cm}
  \begin{center}
  \begin{footnotesize}
    \textsf{Prof. Dr.-Ing. Sebastian Schlesinger} \\[-0.1cm]
    \textsf{Fachbereich 2 - Duales Studium Wirtschaft \& Technik}
    %\url{https://www.hwr-berlin.de}
  \end{footnotesize}
  \end{center}
\end{minipage}
\hfill
%\begin{minipage}[t][1.6cm][c]{3.2cm}
%  \includegraphics[width=2.8cm]{include/ESlogo2}
%\end{minipage}

\vspace*{-0.3cm}
\begin{center} 
  \hrulefill \\[0.1cm]
  {\large Übungsblatt } \blattnummer \\[0.15cm]
  {\huge \bfseries Mathematik I - Theoretische Informatik} \\[0.10cm]
  {HWR Berlin, Wintersemester 2025} \\[-0.4cm]
  \begin{tabular}{lcr}
    \hspace{0.3\textwidth}   & \hspace{0.3\textwidth} & \hspace{0.3\textwidth} \\
    Prof. Dr.-Ing. Sebastian Schlesinger   %&                        & "Ubung zur Klausurvorbereitung    \\ 
    %Blatt \blattnummer     % \abgabetermin          \\ 
  \end{tabular} \\[0.1cm]
  \hrulefill
\end{center}}

% \begin{notes} \small
% 	\textbf{Abgabetermine für Blatt 1:}
	
% 	Aufgaben 1.2/1.3: Montag, 11. April, 09:00 Uhr \\
% 	Aufgabe 1.4: Mittwoch, 20. April, 18:00 Uhr
% \end{notes}


\aufgabetitel{$5$}{Beweis}\\
Seien $A,B,M$ Mengen mit $A,B\subseteq M$. Zeigen Sie \[M\backslash (A\cup B) = (M\backslash A)\cap (M\backslash B)\]

\aufgabetitel{$2$}{Funktionen}\\
Geben Sie f"ur die folgenden Funktionen an, ob sie injektiv oder surjektiv sind und begr"unden Sie Ihre Entscheidung.\\

\begin{enumerate}[(i)]
\item $f:\ZZ\to\NN, x\mapsto |x|$
\item $f:\emptyset\to\{0\}$
\end{enumerate}

\aufgabetitel{$4$}{Indexmengen und Beweis}\\
Es sei $A_i=\{n\in\mathbb{N}\mid n<i\}$.
\begin{enumerate}[(i)]
    \item Bestimmen Sie $A_4$.
    \item Zeigen Sie, dass $\bigcap_{i\in\mathbb{N}}A_i=\emptyset$.
\end{enumerate}

\aufgabetitel{$4$}{Beweis mit Relationen}\\
Seien $R_1,S_1,R_2,S_2\subseteq M\times M$ Relationen auf einer Menge $M$. Zeigen Sie \[R_1\subseteq R_2\wedge S_1\subseteq S_2\Rightarrow R_1\circ S_1\subseteq R_2\circ S_2\]

\aufgabetitel{$5$}{"Aquivalenzrelation}\\
Sei $x\sim y\Leftrightarrow x=y \vee x=-y$ eine Relation auf $\mathbb{Z}\times\mathbb{Z}$. 
\begin{enumerate}[(i)]
    \item Zeigen Sie, dass $\sim$ eine Äquivalenzrelation ist.
    \item Bestimmen Sie die Quotientenmenge $\mathbb{Z}/\sim$.
\end{enumerate}

\aufgabetitel{$6$}{Beweise}\\
Zeigen Sie
\begin{enumerate}[(i)]
    \item $B\backslash (B\backslash A)=A\cap B$ f"ur Mengen $A,B$.
    \item $f^{-1}(B_1\cap B_2)=f^{-1}(B_1)\cap f^{-1}(B_2)$ f"ur Abbildungen $f:A\to B$ und $B_1,B_2\subseteq B$. 
    
    (Hinweis: Es ist $f^{-1}(X)=\{a\in A\mid f(a)\in X\}$.)
\end{enumerate}


\aufgabetitel{$5$}{Mengenbeweise} \\
Beweisen Sie folgende Aussagen:
\begin{enumerate}[(i)]
  \item $A\subseteq B\cap C\Leftrightarrow A\subseteq B\wedge A\subseteq C$
  \item $A\backslash(B\cup C)=(A\backslash B)\cap (A\backslash C)$
  %\item $\bigcap_{n\in\mathbb{N}}\{m\in\mathbb{N}|m\geq n\}=\emptyset$
  \item $(A\cup B)\times C=(A\times C)\cup (B\times C)$
  \item $\left(\bigcup_{i\in I}D_i\right)\cap B=\bigcup_{i\in I}(D_i\cap B)$
  \item $\bigcap_{\varepsilon\in\mathbb{R}\backslash\{0\}}\{x\in\mathbb{R}||x-\pi|\leq |\varepsilon|\}=\{\pi\}$
\end{enumerate}

\aufgabetitel{$4$}{Beweis mit Relationen}\\
Sei $R\subseteq M\times M$ auf einer Menge $M$. Zeigen Sie \[R\mbox{ transitiv }\Leftrightarrow R\circ R\subseteq R\]


\aufgabetitel{$5$}{Funktionen und Relationen}\\
Sei $f:X\to Y$ eine Funktion und $\sim$ eine Relation "uber $X$ mit $x\sim y:\Leftrightarrow f(x)\leq f(y)$.
\begin{enumerate}[(i)]
    \item Stellen Sie die Funktion $f$ und die Relation $\sim$ f"ur $X=\{1,2,3,4,5\}$, $Y=\{1,2,3\}$, $1\mapsto 1, 2\mapsto 1, 3\mapsto 2, 4\mapsto 2, 5\mapsto 3$ dar.
    \item Zeigen Sie: $\sim$ ist eine Ordnungsrelation genau dann, wenn $f$ injektiv ist.
\end{enumerate}



\end{document}
