%!TEX TS-program = pdflatex
%!TEX TS-options = -shell-escape
% % % % %   Die folgenden Zeilen müssen ihre Zeilennummern 4 und 5 behalten !!!    % % % % %
\newcommand{\printpraesenzlsg}{false}
\newcommand{\printloesungen}{false}
\newcommand{\printbewertungen}{false}
% % % % %   \newcommand{\printloesungen}{false}                                    % % % % %
\newcommand{\blattnummer}{5}
%\newcommand{\abgabetermin}{\textcolor{red}{bis 11.04.2022, 09:00 Uhr}}
\input{include/config.tex}
\usepackage{bussproofs}
% Änderungen 2020: Hinweise auf Digitallehre angepasst; Rechnergeschichte entfernt; Blatt 1 und 2 zusammengefasst.
\begin{document}
\iforiginal{\thispagestyle{scrplain}
\vspace*{-3cm}
\begin{minipage}[t][1.1cm][c]{4.5cm}
  \includegraphics[width=4cm]{include/hwr-logo.png}
\end{minipage}
\hfill
\begin{minipage}[t][1.5cm][c]{8cm}
  \begin{center}
  \begin{footnotesize}
    \textsf{Prof. Dr.-Ing. Sebastian Schlesinger} \\[-0.1cm]
    \textsf{Fachbereich 2 - Duales Studium Wirtschaft \& Technik}
    %\url{https://www.hwr-berlin.de}
  \end{footnotesize}
  \end{center}
\end{minipage}
\hfill
%\begin{minipage}[t][1.6cm][c]{3.2cm}
%  \includegraphics[width=2.8cm]{include/ESlogo2}
%\end{minipage}

\vspace*{-0.3cm}
\begin{center} 
  \hrulefill \\[0.1cm]
  {\large Übungsblatt } \blattnummer \\[0.15cm]
  {\huge \bfseries Mathematik I - Theoretische Informatik} \\[0.10cm]
  {HWR Berlin, Wintersemester 2025} \\[-0.4cm]
  \begin{tabular}{lcr}
    \hspace{0.3\textwidth}   & \hspace{0.3\textwidth} & \hspace{0.3\textwidth} \\
    Prof. Dr.-Ing. Sebastian Schlesinger   %&                        & "Ubung zur Klausurvorbereitung    \\ 
    %Blatt \blattnummer     % \abgabetermin          \\ 
  \end{tabular} \\[0.1cm]
  \hrulefill
\end{center}}

% \begin{notes} \small
% 	\textbf{Abgabetermine für Blatt 1:}
	
% 	Aufgaben 1.2/1.3: Montag, 11. April, 09:00 Uhr \\
% 	Aufgabe 1.4: Mittwoch, 20. April, 18:00 Uhr
% \end{notes}



\aufgabetitel{$2$}{Formale Sprache modellieren}\\

Gegeben ist die Sprache $L = \{ w \in \{0,1\}^* \mid w \text{ endet auf } 01 \}$.
Entwerfen Sie einen DFA, der $L$ erkennt.

\aufgabetitel{$3$}{Minimierung von DFA}\\
Gegeben ist folgender DFA $M = (Q, \Sigma, \delta, q_0, F)$ mit:
\begin{itemize}
    \item $Q = \{A, B, C, D, E\}$
    \item $\Sigma = \{0,1\}$
    \item $\delta(A,0)=B,\ \delta(A,1)=C$
    \item $\delta(B,0)=A,\ \delta(B,1)=D$
    \item $\delta(C,0)=E,\ \delta(C,1)=A$
    \item $\delta(D,0)=E,\ \delta(D,1)=B$
    \item $\delta(E,0)=E,\ \delta(E,1)=E$
    \item $q_0 = A$
    \item $F = \{A, D\}$
\end{itemize}
Minimieren Sie den DFA. 

\aufgabetitel{$3$}{Minimierung von DFA}\\
Gegeben ist folgender DFA $M = (Q, \Sigma, \delta, q_0, F)$ mit:
\begin{itemize}
    \item $Q = \{A, B, C, D\}$
    \item $\Sigma = \{0,1\}$
    \item $\delta(A,0)=C,\ \delta(A,1)=B$
    \item $\delta(B,0)=D,\ \delta(B,1)=A$
    \item $\delta(C,0)=A,\ \delta(C,1)=D$
    \item $\delta(D,0)=B,\ \delta(D,1)=C$
    \item $q_0 = A$
    \item $F = \{A,C\}$
\end{itemize}
Minimieren Sie den DFA.


\aufgabetitel{$5$}{DFA für Sprache}\\
Konstruieren Sie einen NFA für die Sprache
$L = \{ w \in \{a,b\}^* \mid \text{ das dritte Zeichen von rechts ist } 1 \}$.

Wandeln Sie den NFA in einen äquivalenten DFA um.


\aufgabetitel{$3$}{Pumping Lemma}\\
Zeigen Sie mit Hilfe des Pumping-Lemmas, dass die Sprache
$L = \{ a^n b^n \mid n\in\mathbb{N}\}$

\aufgabetitel{$5$}{Reguläre Ausdrücke}\\
Gegeben ist der reguläre Ausdruck
$r = (ab^*)^*a$

Geben Sie einen NFA an und einen äquivalenten DFA, die beide die von $r$ beschriebene Sprache erkennen.

\aufgabetitel{$2$}{Grammatik}\\
Gegeben ist die Grammatik $G$:
$S\rightarrow aSb \mid bSa \mid \varepsilon$

Geben Sie die Elemente von $L_4:=\{w\in L(G)\mid |w|\leq 4\}$ an.

Ist L(G) regulär? Begründen Sie.


\end{document}
