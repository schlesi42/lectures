%!TEX TS-program = pdflatex
%!TEX TS-options = -shell-escape
% % % % %   Die folgenden Zeilen müssen ihre Zeilennummern 4 und 5 behalten !!!    % % % % %
\newcommand{\printpraesenzlsg}{false}
\newcommand{\printloesungen}{true}
\newcommand{\printbewertungen}{false}
% % % % %   \newcommand{\printloesungen}{false}                                    % % % % %
\newcommand{\blattnummer}{2}
%\newcommand{\abgabetermin}{\textcolor{red}{bis 11.04.2022, 09:00 Uhr}}
\documentclass[%
  paper=a4,
  fontsize=10pt,
  ngerman
  ]{scrartcl}

% Basics für Codierung und Sprache
% ===========================================================
  \usepackage{scrtime}
  \usepackage{etex}
  \usepackage{shellesc}
  \usepackage[final]{graphicx}
  \usepackage[utf8]{inputenc}
  \usepackage{babel}
  \usepackage[german=quotes]{csquotes}
  \usepackage{datetime2}
  \usepackage{ifthen}
  \usepackage{dingbat}
% ===========================================================

%% Fonts und Typographie
%% ===========================================================
  \usepackage{charter}
  \usepackage{nimbusmononarrow}
  \usepackage[babel=true,final,tracking=smallcaps]{microtype}
  \DisableLigatures{encoding = T1, family = tt* }
  \usepackage{ellipsis}
% ===========================================================

% Farben
% ===========================================================
  \usepackage[usenames,x11names,final,table]{xcolor}
  \definecolor{fbblau}{HTML}{3078AB}
  \definecolor{blackberry}{rgb}{0.53, 0.0, 0.25}
% ===========================================================

% Mathe-Pakete und -Einstellungen
% ===========================================================
  \usepackage{mathtools}
  \usepackage{amssymb}
  \usepackage[bigdelims]{newtxmath}
  \allowdisplaybreaks
  \usepackage{bm}
  \usepackage{wasysym}
% ===========================================================

% TikZ
% ===========================================================
  \usepackage{tikz}
  \usepackage{tikz-cd}
  \usepackage{pgf}
  \usetikzlibrary{matrix}
  \usetikzlibrary{positioning}
  \usetikzlibrary{arrows}
  \usetikzlibrary{shapes}
  \tikzset{>=Latex}
  \usepackage[mode=image]{standalone}
% ===========================================================

% Seitenlayout, Kopf-/Fußzeile
% ===========================================================
  \usepackage{scrlayer-scrpage}
  \usepackage[top=3cm, bottom=2.5cm, left=2.5cm, right=2cm]{geometry}  
  \pagestyle{scrheadings}
  \clearscrheadfoot 
  \setheadsepline{0.4pt}
  \setfootsepline{0.4pt}
  \setkomafont{pagehead}{\normalfont\footnotesize}
  \setkomafont{pagefoot}{\normalfont\footnotesize}
  \lohead[]{Klausur zur Vorlesung \textit{Mathematik I - Theoretische Grundlagen der Informatik} -- Wintersemester 2022/2023}
  \rohead[]{%
    \ifthenelse{\boolean{loesungen}}{\textbf{Lösungen zu Blatt \blattnummer}}{Blatt \blattnummer}}
  %\rofoot[Version: \DTMnow]{%
  %  \ifthenelse{\boolean{loesungen}}{Version: \DTMnow}{}}
  %\lofoot[\jobname.tex]{%
  %  \ifthenelse{\boolean{loesungen}}{\jobname.tex}{}}
  \cfoot[\thepage]{\thepage}
  \raggedbottom
  \usepackage{setspace}          
  \linespread{1.2}
  \setlength{\parindent}{0pt}
  \setlength{\parskip}{0.5\baselineskip}
  \usepackage[all]{nowidow}
  \usepackage{lscape}
% ===========================================================

% Hyperref
% ===========================================================
  \usepackage[%
    hidelinks,
    pdfpagelabels,
    bookmarksopen=true,
    bookmarksnumbered=true,
    linkcolor=black,
    urlcolor=fbblau,
    plainpages=false,
    pagebackref,
    citecolor=black,
    hypertexnames=true,
    pdfborderstyle={/S/U},
    linkbordercolor=fbblau,
    colorlinks=false,
    backref=false]{hyperref}
  \hypersetup{final}
  \makeatletter
  \g@addto@macro{\UrlBreaks}{\UrlOrds}
  \makeatother
% ===========================================================

% Listen und Tabellen
% ===========================================================
  \usepackage{multicol}
  \usepackage{multirow}
  \usepackage[shortlabels]{enumitem}
  \setlist[enumerate]{font=\bfseries,itemsep=0.25\baselineskip}
  \setlist[itemize]{label=$\triangleright$,itemsep=-0.25\baselineskip}
  \usepackage{tabularx}
  \usepackage{array}
  \usepackage{listings}
  \lstset{breaklines=true,
    basicstyle=\ttfamily,
    frame=tblr,
    language=c,
    numbers=left}
  \usepackage{include/mips}

\definecolor{lightgreen}{rgb}{0,0.7,0}
\definecolor{gray}{rgb}{0.5,0.5,0.5}
\definecolor{mauve}{rgb}{0.58,0,0.82}

\lstdefinestyle{mips}{ %
  language=[mips]Assembler,       % the language of the code
  keywordstyle=\color{blue},          % keyword style
  keywordstyle=[2]\color{magenta},     % keyword style
  keywordstyle=[3]\color{red},          % keyword style
  commentstyle=\color{lightgreen},       % comment style
  stringstyle=\color{mauve},         % string literal style
  escapeinside={\%*}{*)},            % if you want to add a comment within your code
  morekeywords={*,...}               % if you want to add more keywords to the set
}

\newcommand{\mipsinline}[1]{\lstinline[style=mips,columns=fixed]{#1}}
\newcommand{\mipsregister}[1]{\mipsinline{\$#1}}
  \newcommand{\qed}{\hfill \ensuremath{\Box}}
  \newcommand{\cmd}[1]{\texttt{#1}}
% ===========================================================

% Sonstiges
% ===========================================================
  \usepackage[textsize=small,color=Red1!80!OrangeRed1!80]{todonotes}
  \usepackage{lipsum}
% ===========================================================

% Shortcuts and Math Commands
% ===========================================================
  \newcommand{\BB}{\mathbb{B}}      % Bool
  \newcommand{\CC}{\mathbb{C}}      % komplexe Zahlen
  \newcommand{\NN}{\mathbb{N}}      % nat. Zahlen
  \newcommand{\QQ}{\mathbb{Q}}      % rat. Zahlen
  \newcommand{\RR}{\mathbb{R}}      % reelle Zahlen
  \newcommand{\ZZ}{\mathbb{Z}}      % ganze Zahlen
  \newcommand{\oh}{\mathcal{O}}      % Landau-O
  \newcommand{\ol}[1]{\overline{#1}}    % überstreichen
  \newcommand{\wt}[1]{\widetilde{#1}}    % überschlängeln
  \newcommand{\wh}[1]{\widehat{#1}}    % überdachen
  \newcommand{\entspricht}{\mathrel{\widehat{=}}}
  
  \DeclarePairedDelimiter{\absolut}{\lvert}{\rvert}    % Betrag
  \DeclarePairedDelimiter{\ceiling}{\lceil}{\rceil}    % aufrunden
  \DeclarePairedDelimiter{\Floor}{\lfloor}{\rfloor}    % aufrunden
  \DeclarePairedDelimiter{\Norm}{\lVert}{\rVert}      % Norm
  \DeclarePairedDelimiter{\sprod}{\langle}{\rangle}    % spitze Klammern
  \DeclarePairedDelimiter{\enbrace}{(}{)}          % runde Klammern
  \DeclarePairedDelimiter{\benbrace}{\lbrack}{\rbrack}  % eckige Klammern
  \DeclarePairedDelimiter{\penbrace}{\{}{\}}        % geschweifte Klammern
  \newcommand{\Underbrace}[2]{{\underbrace{#1}_{#2}}}   % bessere Unterklammerungen
  
  % Kurzschreibweisen für Faule und Code-Vervollständigung
  \newcommand{\abs}[1]{\absolut*{#1}}
  \newcommand{\ceil}[1]{\ceiling*{#1}}
  \newcommand{\flo}[1]{\Floor*{#1}}
  \newcommand{\no}[1]{\Norm*{#1}}
  \newcommand{\sk}[1]{\sprod*{#1}}
  \newcommand{\enb}[1]{\enbrace*{#1}}
  \newcommand{\penb}[1]{\penbrace*{#1}}
  \newcommand{\benb}[1]{\benbrace*{#1}}
  \newcommand{\stack}[2]{\makebox[1cm][c]{$\stackrel{#1}{#2}$}}

% Übungszettel-Krams
% ===========================================================
  \usepackage[tikz]{mdframed}
  \newmdenv[%
    backgroundcolor = gray!20,
    linewidth=0pt,
    skipabove = 0.5cm,
    skipbelow = 0cm
  ]{notes}
  
  % Lösungsausgabe
  \newboolean{praesenzlsgn}
  \setboolean{praesenzlsgn}{\printpraesenzlsg}
  \newboolean{loesungen}
  \setboolean{loesungen}{\printloesungen}
  % Bewertungsschema-Ausgabe
  \newboolean{bewertungen}
  \setboolean{bewertungen}{\printbewertungen}

  \newcommand{\iforiginal}[1]{\ifthenelse{\boolean{praesenzlsgn}}{}{#1}}
  \newcommand{\ifpraesenzlsg}[1]{\ifthenelse{\boolean{praesenzlsgn}}{#1}{}}
  \newcommand{\ifloesung}[1]{\ifthenelse{\boolean{loesungen}}{#1}{}}
  \newcommand{\ifbewertung}[1]{\ifthenelse{\boolean{bewertungen}}{#1}{}}

  \newcommand{\umbruchoriginal}{\iforiginal{\newpage}}        % Seitenumbruch nur im Original
  \newcommand{\umbruchpraesenzlsg}{\ifpraesenzlsg{\newpage}}  % Seitenumbruch nur in Lösung d. Präsenzaufgabe
  \newcommand{\umbruchlsg}{\ifloesung{\newpage}}              % Seitenumbruch nur in Lösung
  \newcommand{\umbruchbewertung}{\ifbewertung{\newpage}}      % Seitenumbruch nur in Lösung mit Bewertungsschema
  \newcommand{\umbruchnurpraesenz}{\ifthenelse{\boolean{loesungen}}{}{\umbruchpraesenzlsg}}
  \newcommand{\umbruchnurlsg}{\ifthenelse{\boolean{bewertungen}}{}{\umbruchlsg}}

  \usepackage{comment}
  \newenvironment{praesenzlsg}%
    {\ifthenelse{\boolean{praesenzlsgn}}{\textbf{--- Präsenzaufgabe Anfang ---} \mbox{} \newline}{}}%
	{\ifthenelse{\boolean{praesenzlsgn}}{\mbox{} \newline \textbf{--- Präsenzaufgabe Ende ---}}{}}
  \newenvironment{loesung}%
    {\ifthenelse{\boolean{loesungen}}{\textbf{--- Lösung Anfang ---} \mbox{} \newline}{}}%
    {\ifthenelse{\boolean{loesungen}}{\mbox{} \newline \textbf{--- Lösung Ende ---} \ifthenelse{\boolean{bewertungen}}{}{\umbruchlsg}}{}}
  \newenvironment{bewertung}%
    {\ifthenelse{\boolean{bewertungen}}{\textbf{--- Bewertungsschema Anfang ---} \mbox{} \newline}{}}%
    {\ifthenelse{\boolean{bewertungen}}{\mbox{} \newline \textbf{--- Bewertungsschema Ende ---} \umbruchlsg}{}}

  \ifthenelse{\boolean{praesenzlsgn}}{}{\excludecomment{praesenzlsg}}
  \ifthenelse{\boolean{loesungen}}{}{\excludecomment{loesung}}
  \ifthenelse{\boolean{bewertungen}}{}{\excludecomment{bewertung}}

  % Aufgabennummerierung
  \newcounter{aufgabennummer} 
  \setcounter{aufgabennummer}{1}
  \newcommand{\aufgabe}[1]{\textbf{Aufgabe \blattnummer.\theaufgabennummer} \stepcounter{aufgabennummer} \hfill (#1 Punkte)}
  %\newcommand{\aufgabetitel}[2]{\iforiginal{\vspace{0.5cm}} \textbf{Aufgabe \blattnummer.\theaufgabennummer \ (#2)} \stepcounter{aufgabennummer} \hfill (#1 Punkte)}
  \newcommand{\aufgabetitel}[2]{\iforiginal{\vspace{0.5cm}} \textbf{Aufgabe \theaufgabennummer \ (#2)} \stepcounter{aufgabennummer} \hfill (#1 Punkte)}
  \newcommand{\praesenztitel}[1]{\iforiginal{\vspace{0.5cm}} \textbf{Aufgabe \blattnummer.\theaufgabennummer \ (#1)} \stepcounter{aufgabennummer} \hfill \smallpencil}

\newcommand{\danger}{\raisebox{0.5mm}{\fontencoding{U}\fontfamily{futs}\selectfont\char 66\relax}}
\newcommand{\ddanger}[1]{{\danger} \textbf{#1} {\danger}}

% Rechnerstrukturen-Krams
% ===========================================================
  \newcommand{\Band}{\ensuremath{\mathbin{\texttt{AND}}}}
  \newcommand{\Bor}{\ensuremath{\mathbin{\texttt{OR}}}}
  \newcommand{\Bnot}{\ensuremath{\mathbin{\texttt{NOT}}}}
  \newcommand{\Bnand}{\ensuremath{\mathbin{\texttt{NAND}}}}
  \newcommand{\Bnor}{\ensuremath{\mathbin{\texttt{NOR}}}}
  \newcommand{\Bxor}{\ensuremath{\mathbin{\texttt{XOR}}}}
  \newcommand{\Bxnor}{\ensuremath{\mathbin{\texttt{XNOR}}}}
  \newcommand{\Bone}{\ensuremath{\mathbin{\texttt{ONE}}}}
  \newcommand{\Bzero}{\ensuremath{\mathbin{\texttt{ZERO}}}}
  \newcommand{\Binary}[1]{\ensuremath{\texttt{#1}}\xspace}
  \newcommand{\ieee}[3]{\fbox{\ensuremath{\texttt{#1}}}\fbox{\ensuremath{\texttt{#2}}}\fbox{\ensuremath{\texttt{#3}}}}
  \DeclareMathOperator{\DNF}{DNF}
  \DeclareMathOperator{\KNF}{KNF}
  \DeclareMathOperator{\xor}{\oplus}
  \newcommand{\dc}{\star} %don't-care-Wert
 

% Änderungen 2020: Hinweise auf Digitallehre angepasst; Rechnergeschichte entfernt; Blatt 1 und 2 zusammengefasst.
\begin{document}
\iforiginal{\thispagestyle{scrplain}
\vspace*{-3cm}
\begin{minipage}[t][1.1cm][c]{4.5cm}
  \includegraphics[width=4cm]{include/hwr-logo.png}
\end{minipage}
\hfill
\begin{minipage}[t][1.5cm][c]{8cm}
  \begin{center}
  \begin{footnotesize}
    \textsf{Prof. Dr.-Ing. Sebastian Schlesinger} \\[-0.1cm]
    \textsf{Fachbereich 2 - Duales Studium Wirtschaft \& Technik}
    %\url{https://www.hwr-berlin.de}
  \end{footnotesize}
  \end{center}
\end{minipage}
\hfill
%\begin{minipage}[t][1.6cm][c]{3.2cm}
%  \includegraphics[width=2.8cm]{include/ESlogo2}
%\end{minipage}

\vspace*{-0.3cm}
\begin{center} 
  \hrulefill \\[0.1cm]
  {\large Klausur} \\[0.15cm]
  {\huge \bfseries Mathematik I - Theoretische Grundlagen der Informatik} \\[0.10cm]
  {HWR Berlin, Wintersemester 2023/2024} \\[-0.4cm]
  \begin{tabular}{lcr}
    \hspace{0.3\textwidth}   & \hspace{0.3\textwidth} & \hspace{0.3\textwidth} \\
    Prof. Dr.-Ing. Sebastian Schlesinger   %&                        & "Ubung zur Klausurvorbereitung    \\ 
    %Blatt \blattnummer     % \abgabetermin          \\ 
  \end{tabular} \\[0.1cm]
  \hrulefill
\end{center}}

% \begin{notes} \small
% 	\textbf{Abgabetermine für Blatt 1:}
	
% 	Aufgaben 1.2/1.3: Montag, 11. April, 09:00 Uhr \\
% 	Aufgabe 1.4: Mittwoch, 20. April, 18:00 Uhr
% \end{notes}

\aufgabetitel{$5$}{Mengenbeweise} \\
Beweisen Sie folgende Aussagen:
\begin{enumerate}[(i)]
  \item $A\subseteq B\cap C\Leftrightarrow A\subseteq B\wedge A\subseteq C$
  \item $A\backslash(B\cup C)=(A\backslash B)\cap (A\backslash C)$
  %\item $\bigcap_{n\in\mathbb{N}}\{m\in\mathbb{N}|m\geq n\}=\emptyset$
  \item $(A\cup B)\times C=(A\times C)\cup (B\times C)$
  \item $\left(\bigcup_{i\in I}D_i\right)\cap B=\bigcup_{i\in I}(D_i\cap B)$
  \item $\bigcap_{\varepsilon\in\mathbb{R}\backslash\{0\}}\{x\in\mathbb{R}||x-\pi|\leq |\varepsilon|\}=\{\pi\}$
\end{enumerate}

\begin{loesung}
  \textbf{Beweis (i):}
  \glqq$\Rightarrow$\grqq:\\
  Sei $A\subseteq B\cap C$. Wir zeigen $A\subseteq B\wedge A\subseteq C$.\\
  Sei $x\in A$. Wir m"ussen zeigen, dass dann $x\in B$ und $x\in C$. Nach Voraussetzung ist $x\in B\cap C$, d.h. $x\in B\wedge x\in C$, was zu zeigen war. \\
  \glqq$\Leftarrow$\grqq: analog
  
  \qed

  \textbf{Beweis (ii):}\\
  Sei $x\in A\backslash (B\cup C)$

  $\Leftrightarrow x\in A\wedge x \notin (B\cup C)$
  
  $\Leftrightarrow x\in A\wedge x \notin B \wedge x\notin C$

  $\Leftrightarrow x\in A\wedge x\notin B\wedge x\in A\wedge x\notin C$

  $\Leftrightarrow x\in A\backslash B\wedge x\in A\backslash C$

  $\Leftrightarrow x\in (A\backslash B)\cap (A\backslash C)$
  
  \qed

  \textbf{Beweis (iii):}\\
  Sei $x\in (A\cup B)\times C$

  $\Leftrightarrow x\in A\times C\vee x\in B\times C$

  $\Leftrightarrow x\in (A\times C)\cup (B\times C)$
  
  \qed

  \textbf{Beweis (iv):}\\
  Sei $x\in (\bigcup_{i\in I}D_i)\cap B$

  $\Leftrightarrow x\in \{x|\exists i\in I:x\in D_i\}\cap B$

  $\Leftrightarrow x\in (\bigcup_{i\in I}D_i\cap B)$
  
  \qed

  \textbf{Beweis (v):}\\
  \glqq $\subseteq$\grqq:\\
  Annahme: $\exists x\in\bigcap_{\varepsilon\in\mathbb{R}\backslash\{0\}}\{x\in\mathbb{R}||x-\pi|\leq |\varepsilon|\}:x\neq\pi$

  Wegen $x\neq\pi$ muss es ein $\varepsilon>0$ geben mit $|x-\pi|>\varepsilon$. Andererseits ist $x\in\bigcap_{\varepsilon>0}\{x\in\mathbb{R}||x-\pi|\leq \varepsilon\}=\{y|\forall\varepsilon>0:|y-\pi|\leq \varepsilon\}$, 
  d.h. $\forall\varepsilon>0:|x-\pi|\leq\varepsilon$. Widerspruch! Also muss $x=\pi$ sein.

  \glqq $\supseteq$\grqq:\\
  Da $|\pi-\pi|=0\leq\varepsilon$ f"ur beliebige (also alle) $\varepsilon>0$ und $\pi\in\mathbb{R}$ folgt, dass $\pi\in\bigcap_{\varepsilon\in\mathbb{R}\backslash\{0\}}\{x\in\mathbb{R}||x-\pi|\leq |\varepsilon|\}$
  
  \qed
\end{loesung}

\aufgabetitel{$4$}{Symmetrische Differenz}\\
  Unter 
  \[A\triangle B:=(A\backslash B)\cup (B\backslash A)\] versteht man die \textit{symmetrische Differenz} der Mengen $A$ und $B$.
  \begin{enumerate}[(i)]
    \item Machen Sie sich anhand eines Venn-Diagramms klar, was unter der symmetrischen Differenz anschaulich zu verstehen ist.
    \item Beweisen Sie: $\forall A,B:A\triangle B=(A\cup B)\backslash(A\cap B)$.
  \end{enumerate}

\begin{loesung}
Venn-Diagramm zeichne ich jetzt mal nicht. Bleibt als "Ubung.

\textbf{Beweis:}\\
Sei $x\in A\triangle B=(A\backslash B)\cup (B\backslash A)$

$\Leftrightarrow (x\in A\wedge x\notin B)\vee (x\in B\wedge x\notin A)$

$\Leftrightarrow (x\in A\vee x\in B)\wedge (x\in A\vee x\notin A)\wedge (x\notin B\vee x\in B)\wedge (x\notin B\vee x\notin A)$ (Distributivit"at)

$\Leftrightarrow (x\in A\vee x\in B)\wedge (x\notin A\vee x\notin B)$ ($x\in A\vee x\notin A$ ist immer wahr, kann man also k"urzen)

$\Leftrightarrow x\in (A\cup B)\wedge \neq (x\in A\wedge x\in B)$ (De Morgan)

$\Leftrightarrow x\in(A\cup B)\backslash(A\cap B)$

\qed
\end{loesung}

\aufgabetitel{$2$}{Beweisen oder Widerlegen} \\
Beweisen oder widerlegen Sie:\\
Aus $A_1\cap A_2\neq\emptyset, A_2\cap A_3\neq\emptyset$ und $A_1\cap A_3\neq\emptyset$ folgt $\bigcap_{i\in\{1.2.3\}}A_i\neq\emptyset$.

\begin{loesung}
  Sei $A_1=\{1,2\}, A_2=\{2,3\},A_3=\{1,3\}$. Dann ist $A_1\cap A_2\neq\emptyset, A_2\cap A_3\neq\emptyset$ und $A_1\cap A_2\neq \emptyset$. 
  Aber $\bigcap_{i\in\{1,2,3\}}A_i=A_1\cap A_2\cap A_3=\emptyset$. 
  
  Die Behauptung ist also falsch.
\end{loesung}
\aufgabetitel{$2$}{Relation}\\
Gegeben sei die Relation $R=\{(a,a),(a,b),(b,c),(c,b),(a,d)\}$.
\begin{enumerate}[(i)]
  \item Stellen Sie die Adjazenzmatrix der Relation dar.
  \item Stellen Sie die Relation als Graph dar.
\end{enumerate}

\begin{loesung}
Das ist eine ganz einfache Aufgabe. Lass ich mal als "Ubung.
\end{loesung}

\aufgabetitel{$3$}{Mengen}\\
Bestimmen Sie die folgenden Mengen:
\begin{enumerate}[(i)]
  \item $(\{1,2\}\times\{3,4\})\cup\{1,2,3\}$
  \item $\{a,b\}\times\mathscr{P}(\{1,2\})$
  \item $\mathscr{P}(\{1,2\})\cap\mathscr{P}(\{1\})$
\end{enumerate}

\begin{loesung}
\begin{enumerate}[(i)]
  \item $\{1,2,3,(1,3),(1,4),(2,3),(2,4)\}$
  \item $\{(a,\emptyset),(b,\emptyset),(a,\{1\}),(b,\{1\}),(a,\{2\}),(b,\{2\}),(a,\{1,2\}),(b,\{1,2\})\}$
  \item $\{\emptyset, \{1\}\}$
  
\end{enumerate}
\end{loesung}

\aufgabetitel{$11$}{Aussagen "uber Mengen}\\
Es sei $A=\{1,2\}$ und $B=\{1,2,3\}$. Welche der folgenden Beziehungen sind richtig?
\begin{enumerate}[(i)]
  \item $1\in A$
  \item $\{1\}\subseteq A$
  \item $1\in\mathscr{P}(A)$
  \item $\{1\}\in\mathscr{P}(A)$
  \item $\mathscr{P}(A)\subseteq\mathscr{P}(B)$
  \item $A\in\mathscr{P}(B)$
  \item $\emptyset\in\mathscr{P}(A)$
  \item $\emptyset\subseteq\mathscr{P}(A)$
  \item $\{\{1\},A\}\subseteq\mathscr{P}(A)$
  \item $(1,2)\in\mathscr{P}(A\times B)$
  \item $\{1,2\}\times\{1,2\}\in\mathscr{P}(A)\times\mathscr{P}(B)$
\end{enumerate}

\begin{loesung}
  \begin{enumerate}[(i)]
    \item wahr
    \item wahr
    \item falsch
    \item wahr
    \item wahr
    \item wahr
    \item wahr
    \item wahr
    \item wahr
    \item wahr
    \item falsch
  \end{enumerate}
\end{loesung}

\aufgabetitel{$4$}{Kartesische Produkte}\\
Es sei $A=\{1,2\}$ und $B=\{2,3,4\}$.
Bilden Sie die folgenden Mengen:
\begin{enumerate}[(i)]
  \item $A\times B$
  \item $(A\times A)\cap (B\times B)$
  \item $(A\times B)\backslash (B\times B)$
  \item $A\times A\times A$
\end{enumerate}

\begin{loesung}
  \begin{enumerate}[(i)]
    \item $A\times B=\{(1,2),(1,3),(1,4),(2,2),(2,3),(2,4)\}$
    \item $(A\times A)\cap (B\times B)=\{(1,1),(1,2),(2,1),(2,2)\}\cap\{(2,2),(2,3),(2,4),(3,2),(3,3),(3,4),(4,2),(4,3),(4,4)\}=\{(2,2)\}$
    \item $(A\times B)\backslash (B\times B)=\{(1,2),(1,3),(1,4)\}$
    \item $A\times A\times A=\{(1,1,1),(1,1,2),(1,2,1),(2,1,1),(1,2,2),(2,1,2),(2,2,1),(2,2,2)\}$
  \end{enumerate}
\end{loesung}

\aufgabetitel{$4$}{Potenzmengenbeweis}\\
Zeigen Sie für beliebige Mengen $A,B$:

  \[A\subseteq B\Leftrightarrow \mathscr{P}(A)\subseteq\mathscr{P}(B)\]


\begin{loesung}
  \glqq$\Rightarrow$\grqq:

  Sei $M\in\mathscr{P}(A)$. Wir wollen zeigen, dass dann auch $M\in\mathscr{P}(B)$.
  
  Dann ist $M\subseteq A$. Da aber $A\subseteq B$ ist auch $M\subseteq B$ und damit $M\in\mathscr{P}(B)$.

  \glqq$\Leftarrow$\grqq:

  Sei $x\in A$. Wir wollen zeigen, dass dann $x\in B$. Es ist $\{x\}\subseteq A$, also $\{x\}\in\mathscr{P}(A)$.
  
  Wegen Voraussetzung ist aber $\mathscr{P}(A)\subseteq\mathscr{P}(B)$, also $\{x\}\in\mathscr{P}(B)$ und damit $x\in B$. 

  \qed
\end{loesung}

\aufgabetitel{$4$}{Potenzmengenbeweis}\\
Zeigen Sie für beliebige Mengen $A,B$:
\[\mathscr{P}(A\cap B)=\mathscr{P}(A)\cap\mathscr{P}(B)\]

\begin{loesung}
Sei $M\in\mathscr{P}(A\cap B)$

$\Leftrightarrow M\subseteq A\cap B$

$\Leftrightarrow \forall x\in M:x\in A\cap B$

$\Leftrightarrow \forall x\in M: x\in A\wedge x\in B$

$\Leftrightarrow M\subseteq A\wedge M\subseteq B$

$\Leftrightarrow M\in\mathscr{P}(A)\wedge M\in\mathscr{P}(B)$

$\Leftrightarrow M\in\mathscr{P}(A)\cap\mathscr{P}(B)$

\qed
\end{loesung}

\aufgabetitel{$2$}{Relationendarstellungen}\\
Sei $R=\{(1,1),(2,2),(1,3),(2,3),(2,1),(3,1)\}$ eine Relation. Stellen Sie die Relation als Graph und Adjazenzmatrix dar.

\aufgabetitel{$5$}{Relation}\\
Diese Aufgabe ist etwas schwieriger.

Wir definieren $a\equiv b\Leftrightarrow 3|(a-b)$ mit $a,b\in\mathbb{Z}$. Beschreiben Sie was die Relation ausdr"uckt.

\begin{notes}
Hinweis: Denken Sie an die Division mit Rest.
\end{notes}

\begin{loesung}
 Zwei Zahlen sind in Relation, wenn sie den gleichen Rest bei der Division durch $3$ haben. 
 
 Begr"undung:

 Jede Zahl $a$ l"asst sich darstellen als $a=3\cdot \xi+\eta$ mit geeignetem $a,\xi,\eta$, wobei $\eta\in\{0,1,2\}$.
 $\eta$ ist der Rest bei der Division von $a$ durch $3$. Hat man nun $b=3\cdot\kappa+\lambda$, dann ist 
 $a-b=3\cdot\xi+\eta-3\cdot\kappa-\lambda=\eta-\lambda$ und es gilt $3|(a-b)\Leftrightarrow \eta-\lambda=0$, also wenn $a$ und $b$ denselben Rest bei der Division durch $3$ ergeben.
\end{loesung}

\end{document}