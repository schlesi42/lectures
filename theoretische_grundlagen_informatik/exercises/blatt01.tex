%!TEX TS-program = pdflatex
%!TEX TS-options = -shell-escape
% % % % %   Die folgenden Zeilen müssen ihre Zeilennummern 4 und 5 behalten !!!    % % % % %
\newcommand{\printpraesenzlsg}{false}
\newcommand{\printloesungen}{false}
\newcommand{\printbewertungen}{false}
% % % % %   \newcommand{\printloesungen}{false}                                    % % % % %
\newcommand{\blattnummer}{1}
%\newcommand{\abgabetermin}{\textcolor{red}{bis 11.04.2022, 09:00 Uhr}}
\input{include/config.tex}

% Änderungen 2020: Hinweise auf Digitallehre angepasst; Rechnergeschichte entfernt; Blatt 1 und 2 zusammengefasst.
\begin{document}
\iforiginal{\thispagestyle{scrplain}
\vspace*{-3cm}
\begin{minipage}[t][1.1cm][c]{4.5cm}
  \includegraphics[width=4cm]{include/hwr-logo.png}
\end{minipage}
\hfill
\begin{minipage}[t][1.5cm][c]{8cm}
  \begin{center}
  \begin{footnotesize}
    \textsf{Prof. Dr.-Ing. Sebastian Schlesinger} \\[-0.1cm]
    \textsf{Fachbereich 2 - Duales Studium Wirtschaft \& Technik}
    %\url{https://www.hwr-berlin.de}
  \end{footnotesize}
  \end{center}
\end{minipage}
\hfill
%\begin{minipage}[t][1.6cm][c]{3.2cm}
%  \includegraphics[width=2.8cm]{include/ESlogo2}
%\end{minipage}

\vspace*{-0.3cm}
\begin{center} 
  \hrulefill \\[0.1cm]
  {\large Übungsblatt } \blattnummer \\[0.15cm]
  {\huge \bfseries Mathematik I - Theoretische Informatik} \\[0.10cm]
  {HWR Berlin, Wintersemester 2025} \\[-0.4cm]
  \begin{tabular}{lcr}
    \hspace{0.3\textwidth}   & \hspace{0.3\textwidth} & \hspace{0.3\textwidth} \\
    Prof. Dr.-Ing. Sebastian Schlesinger   %&                        & "Ubung zur Klausurvorbereitung    \\ 
    %Blatt \blattnummer     % \abgabetermin          \\ 
  \end{tabular} \\[0.1cm]
  \hrulefill
\end{center}}

% \begin{notes} \small
% 	\textbf{Abgabetermine für Blatt 1:}
	
% 	Aufgaben 1.2/1.3: Montag, 11. April, 09:00 Uhr \\
% 	Aufgabe 1.4: Mittwoch, 20. April, 18:00 Uhr
% \end{notes}

\aufgabetitel{$7$}{Aussagen} \\
Sind die folgenden Aussagen im mathematischen Sinne wahr oder
falsch?

\begin{enumerate}[(i)]
  \item Eine un"uberdachte Stra\ss e ist genau dann nass, wenn es geregnet hat.
  \item Wenn $n$ eine Primzahl ist, dann ist $n$ ungerade.
  \item Wenn eine Wand gelb ist, dann ist sie gelb oder gr"un.
  \item Wenn die Erde eine Scheibe ist, dann ist $1=1$.
  \item Wenn $3=4$ ist, dann ist $10=20$.
  \item Jede nat"urliche Zahl ist gr"o\ss er als $10$ oder kleiner als $100$.
  \item Es ist $0=0$ oder $1=1$.
\end{enumerate}


% \aufgabetitel{$5$}{Direkter Beweis} \\
% Zeigen Sie:

% Sei $n$ eine natürliche Zahl. Dann ist $n$ genau dann gerade, wenn es $n^2$ ist.

% \begin{notes}
% 	\textit{Genau dann wenn} bezeichnet eine "Aquivalenz. Zum Beweis zeigen Sie beide Richtungen der Implikation (\glqq $\Rightarrow$\grqq und \glqq$\Leftarrow$\grqq).
% \end{notes}

\aufgabetitel{$3$}{Quantoren} \\
Wir hatten die Quantoren $\forall$ und $\exists$ eingef"uhrt. Mengen wurden noch nicht oder nur teilweise eingef"uhrt. 
Wir holen das hier kurz nach.
\begin{notes}
	Eine Menge ist die Zusammenfassung von bestimmten unterschiedlichen Objekten (die Elemente der Menge) zu einem neuen Ganzen.
	
	Wir schreiben $x\in M$, falls das Objekt $x$ zur Menge $M$ geh"ort.
	Wir schreiben $x\notin M$, falls das Objekt $x$ nicht zur Menge $M$ geh"ort. 
\end{notes}
Man kann demzufolge durch $\forall x\in M: A(x)$ bzw. $\exists x\in M: A(x)$ ausdr"ucken, dass eine Aussage $A(x)$ f"ur alle $x$ aus der Menge $M$ gelten soll bzw.
das es ein $x$ aus $M$ geben soll, so dass $A(x)$ gilt.

Definieren Sie einen neuen Ausdruck \[\exists !x\in M:A(x)\] daf"ur, dass \textit{genau ein} $x$ aus $M$ eine Aussage $A(x)$ erf"ullt!


\aufgabetitel{$6$}{Prädikatenlogik} \\
Gegeben seien folgende Prädikate: 
\begin{enumerate}
  \item $B(x,y): x$ besiegt $y$ 
  \item $F(x): x$ ist Fußball-Nationalmannnschaft
  \item $T(x,y): x$ ist Torwart von $y$
\end{enumerate}
und folgende Konstanten: 
\begin{enumerate}
  \item de: deutsche Nationalmannschaft
  \item us: Nationalmannschaft der USA
  \item br: brasilianische Nationalmannschaft
\end{enumerate} 
Drücken Sie folgende Aussagen in der Prädikatenlogik aus: 
\begin{enumerate}
  \item Jede Fußball-Nationalmannschaft hat einen Torwart.
  \item Wenn de gegen us gewinnt, dann verliert de nicht jedes Spiel.
  \item br schlägt jedes Team, gegen das de verliert, mit Ausnahme von sich selbst.
\end{enumerate} 

\end{document}
