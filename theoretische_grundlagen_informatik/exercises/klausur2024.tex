%!TEX TS-program = pdflatex
%!TEX TS-options = -shell-escape
% % % % %   Die folgenden Zeilen müssen ihre Zeilennummern 4 und 5 behalten !!!    % % % % %
\newcommand{\printpraesenzlsg}{false}
\newcommand{\printloesungen}{false}
\newcommand{\printbewertungen}{false}
% % % % %   \newcommand{\printloesungen}{false}                                    % % % % %
\newcommand{\blattnummer}{1}
%\newcommand{\abgabetermin}{\textcolor{red}{bis 11.04.2022, 09:00 Uhr}}
\input{include/config.tex}

% Änderungen 2020: Hinweise auf Digitallehre angepasst; Rechnergeschichte entfernt; Blatt 1 und 2 zusammengefasst.
\begin{document}

\iforiginal{\thispagestyle{scrplain}
\vspace*{-3cm}
\begin{minipage}[t][1.1cm][c]{4.5cm}
  \includegraphics[width=4cm]{include/hwr-logo.png}
\end{minipage}
\hfill
\begin{minipage}[t][1.5cm][c]{8cm}
  \begin{center}
  \begin{footnotesize}
    \textsf{Prof. Dr.-Ing. Sebastian Schlesinger} \\[-0.1cm]
    \textsf{Fachbereich 2 - Duales Studium Wirtschaft \& Technik}
    %\url{https://www.hwr-berlin.de}
  \end{footnotesize}
  \end{center}
\end{minipage}
\hfill
%\begin{minipage}[t][1.6cm][c]{3.2cm}
%  \includegraphics[width=2.8cm]{include/ESlogo2}
%\end{minipage}

\vspace*{-0.3cm}
\begin{center} 
  \hrulefill \\[0.1cm]
  {\large Übungsblatt } \blattnummer \\[0.15cm]
  {\huge \bfseries Mathematik I - Theoretische Informatik} \\[0.10cm]
  {HWR Berlin, Wintersemester 2025} \\[-0.4cm]
  \begin{tabular}{lcr}
    \hspace{0.3\textwidth}   & \hspace{0.3\textwidth} & \hspace{0.3\textwidth} \\
    Prof. Dr.-Ing. Sebastian Schlesinger   %&                        & "Ubung zur Klausurvorbereitung    \\ 
    %Blatt \blattnummer     % \abgabetermin          \\ 
  \end{tabular} \\[0.1cm]
  \hrulefill
\end{center}}
\aufgabetitel{$4$}{Mengen und Funktionen}\\
Gegeben seien die Mengen $A=\{a,\{a,b\},c\}$ und $B=\{\{c\},a\}$.
\begin{enumerate}[(a)]
\item Geben Sie die Menge $A\cup B$ an.
\item Geben Sie die Menge $A\cap B$ an.
\item Geben Sie die Menge $A\setminus B$ an.
\item Geben Sie $\mathscr{P}(B)$ an.

\end{enumerate}

\aufgabetitel{$10$}{Aussagen zu Mengen}\\
Gegeben sei die Menge $A=\{1,2,\{1\},\{1,2\}\}$, $\mathscr{P}(A)$ die Potenzmenge von $A$. Bewerten Sie die folgenden Aussagen (jeweils ja oder nein angeben).
\begin{enumerate}[(a)]
\item $1\in A$
\item $\{1\}\in A$
\item $\{1\}\subseteq A$
\item $\{\{1\}\}\in A$
\item $\{\{1\}\}\subseteq A$
\item $\{1\}\in\mathscr{P}(A)$
\item $\emptyset\in A$
\item $\emptyset\in\mathscr{P}(A)$
\item $\{\emptyset\}\in\mathscr{P}(A)$
\item $1\in\mathscr{P}(A)$
\end{enumerate}

\aufgabetitel{$6$}{Relationen}\\
Gegeben seien die Relationen $R,S\subseteq\{a,b,c,d\}\times\{a,b,c,d\}$ mit $R=\{(a,b),(b,a),(b,b),(b,c),(d,b)\}$\\und $S=\{(a,c),(a,d),(b,a),(c,b),(d,c)\}$.
\begin{enumerate}[(a)]
    \item Stellen Sie $R$ und $S$ als Graphen und Adjazenzmatrix dar.
    \item Stellen Sie die Relation $R\circ S$ als Graphen dar.
    \item Stellen Sie die Relation $S\circ R$ als Graphen dar.
    \item Zeichnen Sie jeweils f"ur $(R\circ S)^*$ und $(S\circ R)^*$ die Hasse-Diagramme, sofern es Ordnungen sind.\\
    Hinweis: $R^*$ bezeichnet die reflexiv-transitive H"ulle einer Relation $R$.

\end{enumerate}

\aufgabetitel{$8$}{Mengen und Relationen}\\
F"ur die Menge $M=\{1,2,3\}$ seien folgende Relationen $R_1,R_2,R_3$ auf der Potenzmenge von $M$ definiert (also $R_i\subseteq \mathscr{P}(M)\times\mathscr{P}(M)$ f"ur $1\leq i\leq 3$).
\begin{enumerate}
\item $R_1$: \glqq hat die gleiche Anzahl von Elementen wie\grqq
\item $R_2$: \glqq hat weniger Elemente als\grqq
\item $R_3$: \glqq hat kein Element gemeinsam mit\grqq
\end{enumerate}
\begin{enumerate}[(a)]
\item Stellen Sie die Relationen $R_1,R_2$ und $R_3$ dar (Format Ihrer Wahl).
\item Entscheiden Sie, welche Eigenschaften sie haben (einfach nennen, wenn es zutrifft): reflexiv, irreflexiv, symmetrisch, asymmetrisch, antisymmetrisch, transitiv.
\item Falls es Ordnungen sind, geben Sie die Relation an und zeichnen Sie die jeweiligen Hasse-Diagramme.
\item Falls es "Aquivalenzrelationen sind, geben Sie die Relation an und geben f"ur die Relation ein Beispiel f"ur zwei "aquivalente, also in Relation stehende Elemente an.
\end{enumerate}

\aufgabetitel{$5$}{Mengenbeweis}\\
Zeigen Sie, dass f"ur beliebige Mengen $A,B,X$ gilt: \[X\backslash (A\cap B)=X\backslash A\cup X\backslash B\]


\hspace{5cm}
\newpage
\underline{\textbf{Formelsammlung}}\\
Hier eine kleine Formelsammlung. Sie ist nicht vollständig, enthält aber alle wichtigen Statements / Definitionen, die man brauchen könnte.
\begin{enumerate}
\item Aussagen- und Pr"adikatenlogik
\begin{enumerate}
    \item Distributivgesetz: $A\wedge (B \vee C) \Leftrightarrow (A\wedge B) \vee (A\wedge C)$
    \item Distributivgesetz: $A\vee (B \wedge C) \Leftrightarrow (A\vee B) \wedge (A\vee C)$
    \item DeMorgan: $\neg(A\wedge B) \Leftrightarrow \neg A \vee \neg B$
    \item DeMorgan: $\neg(A\vee B) \Leftrightarrow \neg A \wedge \neg B$
    \item Idempotenz: $A\wedge A \Leftrightarrow A$
    \item Idempotenz: $A\vee A \Leftrightarrow A$
    \item $A\wedge \neg A \Leftrightarrow \bot$
    \item $A\vee \neg A \Leftrightarrow \top$
    \item $\neg\neg A \Leftrightarrow A$
    \item $\neg\forall x\in M: A(x) \Leftrightarrow \exists x\in M: \neg A(x)$
    \item $\neg\exists x\in M: A(x) \Leftrightarrow \forall x\in M: \neg A(x)$
\end{enumerate}
\item Mengen
\begin{enumerate}
    \item Teilmenge: $A\subseteq B \Leftrightarrow \forall x\in A: x\in B$
    \item Potenzmenge: $\mathscr{P}(A) = \{B\mid B\subseteq A\}$
    \item Vereinigung: $A\cup B = \{x\mid x\in A \vee x\in B\}$
    \item Schnittmenge: $A\cap B = \{x\mid x\in A \wedge x\in B\}$
    \item Differenzmenge: $A\setminus B = \{x\mid x\in A \wedge x\notin B\}$
    \item Distributivgesetz: $A\cap (B \cup C) \Leftrightarrow (A\cap B) \cup (A\cap C)$
    \item Distributivgesetz: $A\cup (B \cap C) \Leftrightarrow (A\cup B) \cap (A\cup C)$
    \item DeMorgan: $A\setminus (B\cup C) \Leftrightarrow (A\setminus B) \cap (A\setminus C)$
    \item DeMorgan: $A\setminus (B\cap C) \Leftrightarrow (A\setminus B) \cup (A\setminus C)$
    \item Es ist $\bigcup_{i\in I}A_i = \{x\mid \exists i\in I: x\in A_i\}$. 
    \item Es ist $\bigcap_{i\in I}A_i = \{x\mid\forall i\in I: x\in A_i\}$.
\end{enumerate}
\item Relationen
\begin{enumerate}
    \item F"ur Mengen $M,N$ ist $R\subseteq M\times N$ eine Relation von $M$ nach $N$.
    \item $R\subseteq M\times M$ ist reflexiv, wenn $\forall x\in M: (x,x)\in R$.
    \item $R\subseteq M\times M$ ist symmetrisch, wenn $\forall x,y\in M: (x,y)\in R \Rightarrow (y,x)\in R$.
    \item $R\subseteq M\times M$ ist antisymmetrisch, wenn $\forall x,y\in M: (x,y)\in R \wedge (y,x)\in R \Rightarrow x=y$.
    \item $R\subseteq M\times M$ ist transitiv, wenn $\forall x,y,z\in M: (x,y)\in R \wedge (y,z)\in R \Rightarrow (x,z)\in R$.
    \item $R\subseteq M\times M$ ist eine "Aquivalenzrelation, wenn $R$ reflexiv, symmetrisch und transitiv ist.
    \item $R\subseteq M\times M$ ist eine Ordnungsrelation, wenn $R$ reflexiv, antisymmetrisch und transitiv ist.
    \item F"ur eine "Aquivalenzrelation $\sim$ auf $M$ ist $[x]=\{y\in M\mid x\sim y\}$ die "Aquivalenzklasse von $x$, $M/\sim = \{[x]\mid x\in M\}$ die Menge der "Aquivalenzklassen oder Quotientenmenge von $M$ modulo $\sim$. Die Menge der "Aquivalenzklassen ist eine Partition von $M$. Umgekehrt induziert jede Partition eine "Aquivalenzrelation.
    \item F"ur eine Ordnungsrelation $\leq$ auf $M$ und $X\subseteq M$ ist $g$ ein kleinstes Element von $X$, wenn $\forall x\in X: g\leq x$, $g$ ein minimales Element von $X$, wenn $\forall g'\in X: g'\leq g\Rightarrow g=g'$, maximale und gr"o{\ss}te Elemente analog.
    \item Es ist $R^*$ die reflexiv-transitive H"ulle f"ur eine Relation $R$.
\end{enumerate}
\item Funktionen
\begin{enumerate}
    \item Eine Funktion $f:X\to Y$ ist eine Relation (also $f\subseteq X\times Y$), die jedem Element aus der Definitionsmenge $X$ genau ein Element aus der Zielmenge $Y$ zuordnet.
    \item $f$ ist injektiv, wenn $\forall x_1,x_2\in X: f(x_1)=f(x_2) \Rightarrow x_1=x_2$.
    \item $f$ ist surjektiv, wenn $\forall y\in Y: \exists x\in X: f(x)=y$.
    \item $f$ ist bijektiv, wenn $f$ injektiv und surjektiv ist.
    \item Die Umkehrfunktion $f^{-1}$ ist definiert $f^{-1}(Y)=\{x\in X\mid\exists y\in Y: y=f(x)\}$
\end{enumerate}
\end{enumerate}
\end{document}