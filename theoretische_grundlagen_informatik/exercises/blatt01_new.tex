%!TEX TS-program = pdflatex
%!TEX TS-options = -shell-escape
% % % % %   Die folgenden Zeilen müssen ihre Zeilennummern 4 und 5 behalten !!!    % % % % %
\newcommand{\printpraesenzlsg}{false}
\newcommand{\printloesungen}{false}
\newcommand{\printbewertungen}{false}
% % % % %   \newcommand{\printloesungen}{false}                                    % % % % %
\newcommand{\blattnummer}{1}
%\newcommand{\abgabetermin}{\textcolor{red}{bis 11.04.2022, 09:00 Uhr}}
\input{include/config.tex}

% Änderungen 2020: Hinweise auf Digitallehre angepasst; Rechnergeschichte entfernt; Blatt 1 und 2 zusammengefasst.
\begin{document}
\iforiginal{\thispagestyle{scrplain}
\vspace*{-3cm}
\begin{minipage}[t][1.1cm][c]{4.5cm}
  \includegraphics[width=4cm]{include/hwr-logo.png}
\end{minipage}
\hfill
\begin{minipage}[t][1.5cm][c]{8cm}
  \begin{center}
  \begin{footnotesize}
    \textsf{Prof. Dr.-Ing. Sebastian Schlesinger} \\[-0.1cm]
    \textsf{Fachbereich 2 - Duales Studium Wirtschaft \& Technik}
    %\url{https://www.hwr-berlin.de}
  \end{footnotesize}
  \end{center}
\end{minipage}
\hfill
%\begin{minipage}[t][1.6cm][c]{3.2cm}
%  \includegraphics[width=2.8cm]{include/ESlogo2}
%\end{minipage}

\vspace*{-0.3cm}
\begin{center} 
  \hrulefill \\[0.1cm]
  {\large Übungsblatt } \blattnummer \\[0.15cm]
  {\huge \bfseries Mathematik I - Theoretische Informatik} \\[0.10cm]
  {HWR Berlin, Wintersemester 2025} \\[-0.4cm]
  \begin{tabular}{lcr}
    \hspace{0.3\textwidth}   & \hspace{0.3\textwidth} & \hspace{0.3\textwidth} \\
    Prof. Dr.-Ing. Sebastian Schlesinger   %&                        & "Ubung zur Klausurvorbereitung    \\ 
    %Blatt \blattnummer     % \abgabetermin          \\ 
  \end{tabular} \\[0.1cm]
  \hrulefill
\end{center}}

% \begin{notes} \small
% 	\textbf{Abgabetermine für Blatt 1:}
	
% 	Aufgaben 1.2/1.3: Montag, 11. April, 09:00 Uhr \\
% 	Aufgabe 1.4: Mittwoch, 20. April, 18:00 Uhr
% \end{notes}


\aufgabetitel{$9$}{Mengen} \\
Bestimmen Sie die folgenden Mengen:
\begin{enumerate}[(i)]
  \item $\{a,b,c\}\cup \{b,c,d\}$
  \item $\{a,b,c\}\cap \{b,c,d\}$
  \item $\{a,b,c\}\backslash \{b,c,d\}$
  \item $\mathscr{P}(\{1,a\})$
  \item $\mathscr{P}(\{1,\{1\}\})$
  \item $\mathscr{P}(\{1,2,3\})\backslash\mathscr{P}(\{1,2\})$
  \item $\bigcap_{i\in\{2,6\}}\{\frac{i}{2},i+1\}$ (Hinweis: $\bigcap_{i\in I}A_i=\{x|\forall i\in I:x\in A_i\}$ f"ur eine Indexmenge $I$)
  \item $\bigcup_{n\in\mathbb{N}}\{n,n+1,2n\}$ (Hinweis: $\bigcup_{i\in I}A_i=\{x|\exists i\in I:x\in A_i\}$ f"ur eine Indexmenge $I$)
  \item $\mathscr{P}(\mathscr{P}(\mathscr{P}(\emptyset)))$
\end{enumerate}

\aufgabetitel{$6$}{Prädikatenlogik} \\
Gegeben seien folgende Prädikate: 
\begin{enumerate}
  \item $B(x,y): x$ besiegt $y$ 
  \item $F(x): x$ ist Fußball-Nationalmannnschaft
  \item $T(x,y): x$ ist Torwart von $y$
\end{enumerate}
und folgende Konstanten: 
\begin{enumerate}
  \item de: deutsche Nationalmannschaft
  \item us: Nationalmannschaft der USA
  \item br: brasilianische Nationalmannschaft
\end{enumerate} 
Drücken Sie folgende Aussagen in der Prädikatenlogik aus: 
\begin{enumerate}
  \item Jede Fußball-Nationalmannschaft hat einen Torwart.
  \item Wenn de gegen us gewinnt, dann verliert de nicht jedes Spiel.
  \item br schlägt jedes Team, gegen das de verliert, mit Ausnahme von sich selbst.
\end{enumerate} 
\end{document}
