%!TEX TS-program = pdflatex
%!TEX TS-options = -shell-escape
% % % % %   Die folgenden Zeilen müssen ihre Zeilennummern 4 und 5 behalten !!!    % % % % %
\newcommand{\printpraesenzlsg}{false}
\newcommand{\printloesungen}{false}
\newcommand{\printbewertungen}{false}
% % % % %   \newcommand{\printloesungen}{false}                                    % % % % %
\newcommand{\blattnummer}{1}
%\newcommand{\abgabetermin}{\textcolor{red}{bis 11.04.2022, 09:00 Uhr}}
\documentclass[%
  paper=a4,
  fontsize=10pt,
  ngerman
  ]{scrartcl}

% Basics für Codierung und Sprache
% ===========================================================
  \usepackage{scrtime}
  \usepackage{etex}
  \usepackage{shellesc}
  \usepackage[final]{graphicx}
  \usepackage[utf8]{inputenc}
  \usepackage{babel}
  \usepackage[german=quotes]{csquotes}
  \usepackage{datetime2}
  \usepackage{ifthen}
  \usepackage{dingbat}
% ===========================================================

%% Fonts und Typographie
%% ===========================================================
  \usepackage{charter}
  \usepackage{nimbusmononarrow}
  \usepackage[babel=true,final,tracking=smallcaps]{microtype}
  \DisableLigatures{encoding = T1, family = tt* }
  \usepackage{ellipsis}
% ===========================================================

% Farben
% ===========================================================
  \usepackage[usenames,x11names,final,table]{xcolor}
  \definecolor{fbblau}{HTML}{3078AB}
  \definecolor{blackberry}{rgb}{0.53, 0.0, 0.25}
% ===========================================================

% Mathe-Pakete und -Einstellungen
% ===========================================================
  \usepackage{mathtools}
  \usepackage{amssymb}
  \usepackage[bigdelims]{newtxmath}
  \allowdisplaybreaks
  \usepackage{bm}
  \usepackage{wasysym}
% ===========================================================

% TikZ
% ===========================================================
  \usepackage{tikz}
  \usepackage{tikz-cd}
  \usepackage{pgf}
  \usetikzlibrary{matrix}
  \usetikzlibrary{positioning}
  \usetikzlibrary{arrows}
  \usetikzlibrary{shapes}
  \tikzset{>=Latex}
  \usepackage[mode=image]{standalone}
% ===========================================================

% Seitenlayout, Kopf-/Fußzeile
% ===========================================================
  \usepackage{scrlayer-scrpage}
  \usepackage[top=3cm, bottom=2.5cm, left=2.5cm, right=2cm]{geometry}  
  \pagestyle{scrheadings}
  \clearscrheadfoot 
  \setheadsepline{0.4pt}
  \setfootsepline{0.4pt}
  \setkomafont{pagehead}{\normalfont\footnotesize}
  \setkomafont{pagefoot}{\normalfont\footnotesize}
  \lohead[]{Klausur zur Vorlesung \textit{Mathematik I - Theoretische Grundlagen der Informatik} -- Wintersemester 2022/2023}
  \rohead[]{%
    \ifthenelse{\boolean{loesungen}}{\textbf{Lösungen zu Blatt \blattnummer}}{Blatt \blattnummer}}
  %\rofoot[Version: \DTMnow]{%
  %  \ifthenelse{\boolean{loesungen}}{Version: \DTMnow}{}}
  %\lofoot[\jobname.tex]{%
  %  \ifthenelse{\boolean{loesungen}}{\jobname.tex}{}}
  \cfoot[\thepage]{\thepage}
  \raggedbottom
  \usepackage{setspace}          
  \linespread{1.2}
  \setlength{\parindent}{0pt}
  \setlength{\parskip}{0.5\baselineskip}
  \usepackage[all]{nowidow}
  \usepackage{lscape}
% ===========================================================

% Hyperref
% ===========================================================
  \usepackage[%
    hidelinks,
    pdfpagelabels,
    bookmarksopen=true,
    bookmarksnumbered=true,
    linkcolor=black,
    urlcolor=fbblau,
    plainpages=false,
    pagebackref,
    citecolor=black,
    hypertexnames=true,
    pdfborderstyle={/S/U},
    linkbordercolor=fbblau,
    colorlinks=false,
    backref=false]{hyperref}
  \hypersetup{final}
  \makeatletter
  \g@addto@macro{\UrlBreaks}{\UrlOrds}
  \makeatother
% ===========================================================

% Listen und Tabellen
% ===========================================================
  \usepackage{multicol}
  \usepackage{multirow}
  \usepackage[shortlabels]{enumitem}
  \setlist[enumerate]{font=\bfseries,itemsep=0.25\baselineskip}
  \setlist[itemize]{label=$\triangleright$,itemsep=-0.25\baselineskip}
  \usepackage{tabularx}
  \usepackage{array}
  \usepackage{listings}
  \lstset{breaklines=true,
    basicstyle=\ttfamily,
    frame=tblr,
    language=c,
    numbers=left}
  \usepackage{include/mips}

\definecolor{lightgreen}{rgb}{0,0.7,0}
\definecolor{gray}{rgb}{0.5,0.5,0.5}
\definecolor{mauve}{rgb}{0.58,0,0.82}

\lstdefinestyle{mips}{ %
  language=[mips]Assembler,       % the language of the code
  keywordstyle=\color{blue},          % keyword style
  keywordstyle=[2]\color{magenta},     % keyword style
  keywordstyle=[3]\color{red},          % keyword style
  commentstyle=\color{lightgreen},       % comment style
  stringstyle=\color{mauve},         % string literal style
  escapeinside={\%*}{*)},            % if you want to add a comment within your code
  morekeywords={*,...}               % if you want to add more keywords to the set
}

\newcommand{\mipsinline}[1]{\lstinline[style=mips,columns=fixed]{#1}}
\newcommand{\mipsregister}[1]{\mipsinline{\$#1}}
  \newcommand{\qed}{\hfill \ensuremath{\Box}}
  \newcommand{\cmd}[1]{\texttt{#1}}
% ===========================================================

% Sonstiges
% ===========================================================
  \usepackage[textsize=small,color=Red1!80!OrangeRed1!80]{todonotes}
  \usepackage{lipsum}
% ===========================================================

% Shortcuts and Math Commands
% ===========================================================
  \newcommand{\BB}{\mathbb{B}}      % Bool
  \newcommand{\CC}{\mathbb{C}}      % komplexe Zahlen
  \newcommand{\NN}{\mathbb{N}}      % nat. Zahlen
  \newcommand{\QQ}{\mathbb{Q}}      % rat. Zahlen
  \newcommand{\RR}{\mathbb{R}}      % reelle Zahlen
  \newcommand{\ZZ}{\mathbb{Z}}      % ganze Zahlen
  \newcommand{\oh}{\mathcal{O}}      % Landau-O
  \newcommand{\ol}[1]{\overline{#1}}    % überstreichen
  \newcommand{\wt}[1]{\widetilde{#1}}    % überschlängeln
  \newcommand{\wh}[1]{\widehat{#1}}    % überdachen
  \newcommand{\entspricht}{\mathrel{\widehat{=}}}
  
  \DeclarePairedDelimiter{\absolut}{\lvert}{\rvert}    % Betrag
  \DeclarePairedDelimiter{\ceiling}{\lceil}{\rceil}    % aufrunden
  \DeclarePairedDelimiter{\Floor}{\lfloor}{\rfloor}    % aufrunden
  \DeclarePairedDelimiter{\Norm}{\lVert}{\rVert}      % Norm
  \DeclarePairedDelimiter{\sprod}{\langle}{\rangle}    % spitze Klammern
  \DeclarePairedDelimiter{\enbrace}{(}{)}          % runde Klammern
  \DeclarePairedDelimiter{\benbrace}{\lbrack}{\rbrack}  % eckige Klammern
  \DeclarePairedDelimiter{\penbrace}{\{}{\}}        % geschweifte Klammern
  \newcommand{\Underbrace}[2]{{\underbrace{#1}_{#2}}}   % bessere Unterklammerungen
  
  % Kurzschreibweisen für Faule und Code-Vervollständigung
  \newcommand{\abs}[1]{\absolut*{#1}}
  \newcommand{\ceil}[1]{\ceiling*{#1}}
  \newcommand{\flo}[1]{\Floor*{#1}}
  \newcommand{\no}[1]{\Norm*{#1}}
  \newcommand{\sk}[1]{\sprod*{#1}}
  \newcommand{\enb}[1]{\enbrace*{#1}}
  \newcommand{\penb}[1]{\penbrace*{#1}}
  \newcommand{\benb}[1]{\benbrace*{#1}}
  \newcommand{\stack}[2]{\makebox[1cm][c]{$\stackrel{#1}{#2}$}}

% Übungszettel-Krams
% ===========================================================
  \usepackage[tikz]{mdframed}
  \newmdenv[%
    backgroundcolor = gray!20,
    linewidth=0pt,
    skipabove = 0.5cm,
    skipbelow = 0cm
  ]{notes}
  
  % Lösungsausgabe
  \newboolean{praesenzlsgn}
  \setboolean{praesenzlsgn}{\printpraesenzlsg}
  \newboolean{loesungen}
  \setboolean{loesungen}{\printloesungen}
  % Bewertungsschema-Ausgabe
  \newboolean{bewertungen}
  \setboolean{bewertungen}{\printbewertungen}

  \newcommand{\iforiginal}[1]{\ifthenelse{\boolean{praesenzlsgn}}{}{#1}}
  \newcommand{\ifpraesenzlsg}[1]{\ifthenelse{\boolean{praesenzlsgn}}{#1}{}}
  \newcommand{\ifloesung}[1]{\ifthenelse{\boolean{loesungen}}{#1}{}}
  \newcommand{\ifbewertung}[1]{\ifthenelse{\boolean{bewertungen}}{#1}{}}

  \newcommand{\umbruchoriginal}{\iforiginal{\newpage}}        % Seitenumbruch nur im Original
  \newcommand{\umbruchpraesenzlsg}{\ifpraesenzlsg{\newpage}}  % Seitenumbruch nur in Lösung d. Präsenzaufgabe
  \newcommand{\umbruchlsg}{\ifloesung{\newpage}}              % Seitenumbruch nur in Lösung
  \newcommand{\umbruchbewertung}{\ifbewertung{\newpage}}      % Seitenumbruch nur in Lösung mit Bewertungsschema
  \newcommand{\umbruchnurpraesenz}{\ifthenelse{\boolean{loesungen}}{}{\umbruchpraesenzlsg}}
  \newcommand{\umbruchnurlsg}{\ifthenelse{\boolean{bewertungen}}{}{\umbruchlsg}}

  \usepackage{comment}
  \newenvironment{praesenzlsg}%
    {\ifthenelse{\boolean{praesenzlsgn}}{\textbf{--- Präsenzaufgabe Anfang ---} \mbox{} \newline}{}}%
	{\ifthenelse{\boolean{praesenzlsgn}}{\mbox{} \newline \textbf{--- Präsenzaufgabe Ende ---}}{}}
  \newenvironment{loesung}%
    {\ifthenelse{\boolean{loesungen}}{\textbf{--- Lösung Anfang ---} \mbox{} \newline}{}}%
    {\ifthenelse{\boolean{loesungen}}{\mbox{} \newline \textbf{--- Lösung Ende ---} \ifthenelse{\boolean{bewertungen}}{}{\umbruchlsg}}{}}
  \newenvironment{bewertung}%
    {\ifthenelse{\boolean{bewertungen}}{\textbf{--- Bewertungsschema Anfang ---} \mbox{} \newline}{}}%
    {\ifthenelse{\boolean{bewertungen}}{\mbox{} \newline \textbf{--- Bewertungsschema Ende ---} \umbruchlsg}{}}

  \ifthenelse{\boolean{praesenzlsgn}}{}{\excludecomment{praesenzlsg}}
  \ifthenelse{\boolean{loesungen}}{}{\excludecomment{loesung}}
  \ifthenelse{\boolean{bewertungen}}{}{\excludecomment{bewertung}}

  % Aufgabennummerierung
  \newcounter{aufgabennummer} 
  \setcounter{aufgabennummer}{1}
  \newcommand{\aufgabe}[1]{\textbf{Aufgabe \blattnummer.\theaufgabennummer} \stepcounter{aufgabennummer} \hfill (#1 Punkte)}
  %\newcommand{\aufgabetitel}[2]{\iforiginal{\vspace{0.5cm}} \textbf{Aufgabe \blattnummer.\theaufgabennummer \ (#2)} \stepcounter{aufgabennummer} \hfill (#1 Punkte)}
  \newcommand{\aufgabetitel}[2]{\iforiginal{\vspace{0.5cm}} \textbf{Aufgabe \theaufgabennummer \ (#2)} \stepcounter{aufgabennummer} \hfill (#1 Punkte)}
  \newcommand{\praesenztitel}[1]{\iforiginal{\vspace{0.5cm}} \textbf{Aufgabe \blattnummer.\theaufgabennummer \ (#1)} \stepcounter{aufgabennummer} \hfill \smallpencil}

\newcommand{\danger}{\raisebox{0.5mm}{\fontencoding{U}\fontfamily{futs}\selectfont\char 66\relax}}
\newcommand{\ddanger}[1]{{\danger} \textbf{#1} {\danger}}

% Rechnerstrukturen-Krams
% ===========================================================
  \newcommand{\Band}{\ensuremath{\mathbin{\texttt{AND}}}}
  \newcommand{\Bor}{\ensuremath{\mathbin{\texttt{OR}}}}
  \newcommand{\Bnot}{\ensuremath{\mathbin{\texttt{NOT}}}}
  \newcommand{\Bnand}{\ensuremath{\mathbin{\texttt{NAND}}}}
  \newcommand{\Bnor}{\ensuremath{\mathbin{\texttt{NOR}}}}
  \newcommand{\Bxor}{\ensuremath{\mathbin{\texttt{XOR}}}}
  \newcommand{\Bxnor}{\ensuremath{\mathbin{\texttt{XNOR}}}}
  \newcommand{\Bone}{\ensuremath{\mathbin{\texttt{ONE}}}}
  \newcommand{\Bzero}{\ensuremath{\mathbin{\texttt{ZERO}}}}
  \newcommand{\Binary}[1]{\ensuremath{\texttt{#1}}\xspace}
  \newcommand{\ieee}[3]{\fbox{\ensuremath{\texttt{#1}}}\fbox{\ensuremath{\texttt{#2}}}\fbox{\ensuremath{\texttt{#3}}}}
  \DeclareMathOperator{\DNF}{DNF}
  \DeclareMathOperator{\KNF}{KNF}
  \DeclareMathOperator{\xor}{\oplus}
  \newcommand{\dc}{\star} %don't-care-Wert
 

% Änderungen 2020: Hinweise auf Digitallehre angepasst; Rechnergeschichte entfernt; Blatt 1 und 2 zusammengefasst.
\begin{document}

\iforiginal{\thispagestyle{scrplain}
\vspace*{-3cm}
\begin{minipage}[t][1.1cm][c]{4.5cm}
  \includegraphics[width=4cm]{include/hwr-logo.png}
\end{minipage}
\hfill
\begin{minipage}[t][1.5cm][c]{8cm}
  \begin{center}
  \begin{footnotesize}
    \textsf{Prof. Dr.-Ing. Sebastian Schlesinger} \\[-0.1cm]
    \textsf{Fachbereich 2 - Duales Studium Wirtschaft \& Technik}
    %\url{https://www.hwr-berlin.de}
  \end{footnotesize}
  \end{center}
\end{minipage}
\hfill
%\begin{minipage}[t][1.6cm][c]{3.2cm}
%  \includegraphics[width=2.8cm]{include/ESlogo2}
%\end{minipage}

\vspace*{-0.3cm}
\begin{center} 
  \hrulefill \\[0.1cm]
  {\large Klausur} \\[0.15cm]
  {\huge \bfseries Mathematik I - Theoretische Grundlagen der Informatik} \\[0.10cm]
  {HWR Berlin, Wintersemester 2023/2024} \\[-0.4cm]
  \begin{tabular}{lcr}
    \hspace{0.3\textwidth}   & \hspace{0.3\textwidth} & \hspace{0.3\textwidth} \\
    Prof. Dr.-Ing. Sebastian Schlesinger   %&                        & "Ubung zur Klausurvorbereitung    \\ 
    %Blatt \blattnummer     % \abgabetermin          \\ 
  \end{tabular} \\[0.1cm]
  \hrulefill
\end{center}}
\aufgabetitel{$4$}{Mengenoperationen}\\
Gegeben seien die Mengen $A=\{1,2,3,4\}$ und $B=\{1,2,a,b\}$.
\begin{enumerate}[(i)]
    \item Bestimmen Sie $A\cap B$.
    \item Bestimmen Sie $A\cup B$.
    \item Bestimmen Sie $A\backslash B$.
    \item Bestimmen Sie $\mathscr{P}(A\cap B)$
    
\end{enumerate}

\aufgabetitel{$2$}{"Uberlegungen mit Mengen}\\
Die beiden folgenden Aussagen gelten nicht allgemein f"ur alle Mengen $A,B,C$. Finden Sie Gegenbeispiele f"ur Mengen $A,B,C$ zu folgenden Aussagen:
\begin{enumerate}[(i)]
\item $A\cup B\subseteq A\cap B$
\item $A\subseteq B\cup C\Rightarrow A\subseteq B\vee A\subseteq C$
\end{enumerate}

\begin{loesung}
\begin{enumerate}
\item $A=\emptyset, B=\{1\}$
\item $A=\{1,2\},B=\{1\}, C=\{2\}$
\end{enumerate}
\end{loesung}

\aufgabetitel{$2$}{Mengenbeweis}\\
Seien $M,N$ Mengen. Zeigen Sie, dass folgendes gilt:
\[\mathscr{P}(M)\cup\mathscr{P}(N)\subseteq \mathscr{P}(M\cup N)\]


\aufgabetitel{$14$}{Relationen}\\
Gegeben seien die Relation $R\subseteq M\times M$ und $S\subseteq M\times M$ mit $M=\{1,2,3,4\}$, 

$R=\{(1,1),(1,2),(1,4),(2,1),(2,3),(3,2),(4,3)\}$ und $S=\{(1,2),(2,2),(1,4)\}$.
\begin{enumerate}[(i)]
\item Stellen Sie $R$ und $S$ als Adjazenzmatrix und Graph dar.
\item Sind $R$ oder $S$ Ordnungen? Begr"unden Sie Ihre Entscheidung.\\
\textit{(Hinweis zur Erinnerung: Eine Ordnung ist eine Relation, die reflexiv, antisymmetrisch und transitiv ist. Sie muss nicht unbedingt total sein (Total w"urde bedeuten, dass jeweils zwei Elemente immer in Relation stehen - in der einen oder anderen Richtung, also $\forall x,y\in M:(x,y)\in M\vee (y.x)\in M$, was wir in der Vorlesung und auch hier nicht fordern.)).}
\item Berechnen Sie $R\circ S$ und stellen Sie das Ergebnis als Adjazenzmatrix und Graph dar.\\
\textit{(Hinweis zur Erinnerung: Es ist $R\circ S=\{(x,y)\in M\times M|\exists z\in M: (x,z)\in R\wedge (z,y)\in S\}$)}
\item Wir bezeichnen im Folgenden $T=R\circ S$ ($T$ als Abk"urzung von $R\circ S$). Berechnen Sie $T\circ T$ und stellen Sie diese Relation als Adjazenzmatrix und Graph dar.
\item Sei $Id=\{(1,1),(2,2),(3,3),(4,4)\}$ die Relation, die jedes Element von $M$ mit sich selbst in Beziehung setzt. Geben Sie die Adjazenzmatrix und den Graph der Relation $O=Id\cup T\cup T\circ T$ an.
\item Begr"unden Sie, dass $O$ eine Ordnung ist.
\item Stellen Sie $O$ als Hasse-Diagramm dar.

\end{enumerate}

\aufgabetitel{$5$}{Relationen und Eigenschaften}\\
Entscheiden Sie f"ur jede Aussage, ob sie zutrifft, begr"unden Sie Ihre Entscheidung und geben Sie ggf. ein Beispiel an.

Es gibt eine Relation $R\subseteq M\times M$ "uber einer Menge $M$, die 
\begin{enumerate}[(i)]
\item reflexiv und irreflexiv ist\\
\textit{(Hinweis zur Erinnerung: $R$ ist reflexiv, wenn $\forall x\in M:(x,x)\in R$, irreflexiv, wenn $\forall x\in M:(x,x)\notin R$)}
\item weder reflexiv noch irreflexiv ist
\item symmetrisch und antisymmetrisch ist\\
\textit{(Hinweis zur Erinnerung: $R$ ist symmetrisch, wenn $\forall x,y\in M:(x,y)\in R\Rightarrow (y,x)\in R$, \\antisymmetrisch, wenn $\forall x,y\in M:(x,y)\in R\wedge (y,x)\in R\Rightarrow x=y$)}
\item antisymmetrisch und irreflexiv ist
\item reflexiv, symmetrisch und transitiv ist\\
\textit{(Hinweis zur Erinnerung: $R$ ist transitiv, wenn $\forall x,y,z\in M:(x,y)\in R\wedge (y,z)\in R\Rightarrow (x,z)\in R$)}
\end{enumerate}

\begin{loesung}
\begin{enumerate}[(i)]
\item geht nicht, weil sich die Bedingungen ausschlie"sen.
\item $R=\{(1,1)\}$ auf $M=\{1,2\}$
\item $R=\emptyset$
\item $R=\emptyset$
\item "Aquivalenzrelation
\end{enumerate}
\end{loesung}
\aufgabetitel{$4$}{Beweis mit Relationen}\\
Sei $R\subseteq M\times M$ auf einer Menge $M$. Zeigen Sie \[R\mbox{ transitiv }\Leftrightarrow R\circ R\subseteq R\]

\aufgabetitel{$2$}{Funktionen}\\
Geben Sie f"ur die folgenden Funktionen an, ob sie injektiv oder surjektiv sind und begr"unden Sie Ihre Entscheidung.\\
\textit{(Hinweis zur Erinnerung:\\$f:M\to N$ ist injektiv, wenn $\forall x,y\in M:f(x)=f(y)\Rightarrow x=y$ und surjektiv, wenn $\forall y\in N\exists x\in M:y=f(x)$)}

\begin{enumerate}[(i)]
\item $f:\ZZ\to\NN, x\mapsto |x|$
\item $f:\emptyset\to\{0\}$
\end{enumerate}

\end{document}