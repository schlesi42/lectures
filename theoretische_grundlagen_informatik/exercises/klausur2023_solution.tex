%!TEX TS-program = pdflatex
%!TEX TS-options = -shell-escape
% % % % %   Die folgenden Zeilen müssen ihre Zeilennummern 4 und 5 behalten !!!    % % % % %
\newcommand{\printpraesenzlsg}{false}
\newcommand{\printloesungen}{true}
\newcommand{\printbewertungen}{false}
% % % % %   \newcommand{\printloesungen}{false}                                    % % % % %
\newcommand{\blattnummer}{1}
%\newcommand{\abgabetermin}{\textcolor{red}{bis 11.04.2022, 09:00 Uhr}}
\input{include/config.tex}

% Änderungen 2020: Hinweise auf Digitallehre angepasst; Rechnergeschichte entfernt; Blatt 1 und 2 zusammengefasst.
\begin{document}

\iforiginal{\thispagestyle{scrplain}
\vspace*{-3cm}
\begin{minipage}[t][1.1cm][c]{4.5cm}
  \includegraphics[width=4cm]{include/hwr-logo.png}
\end{minipage}
\hfill
\begin{minipage}[t][1.5cm][c]{8cm}
  \begin{center}
  \begin{footnotesize}
    \textsf{Prof. Dr.-Ing. Sebastian Schlesinger} \\[-0.1cm]
    \textsf{Fachbereich 2 - Duales Studium Wirtschaft \& Technik}
    %\url{https://www.hwr-berlin.de}
  \end{footnotesize}
  \end{center}
\end{minipage}
\hfill
%\begin{minipage}[t][1.6cm][c]{3.2cm}
%  \includegraphics[width=2.8cm]{include/ESlogo2}
%\end{minipage}

\vspace*{-0.3cm}
\begin{center} 
  \hrulefill \\[0.1cm]
  {\large Übungsblatt } \blattnummer \\[0.15cm]
  {\huge \bfseries Mathematik I - Theoretische Informatik} \\[0.10cm]
  {HWR Berlin, Wintersemester 2025} \\[-0.4cm]
  \begin{tabular}{lcr}
    \hspace{0.3\textwidth}   & \hspace{0.3\textwidth} & \hspace{0.3\textwidth} \\
    Prof. Dr.-Ing. Sebastian Schlesinger   %&                        & "Ubung zur Klausurvorbereitung    \\ 
    %Blatt \blattnummer     % \abgabetermin          \\ 
  \end{tabular} \\[0.1cm]
  \hrulefill
\end{center}}
\aufgabetitel{$6$}{Mengen und Funktionen}\\
Gegeben seien die Mengen $A=\{1,2,3,\{4\}\}$, $B=\{2,4\}$ und $C=\{\emptyset,\{1\}\}$.
\begin{enumerate}
\item Geben Sie die Menge $A\cup B$ an
\item Geben Sie die Menge $A\cap B$ an
\item Geben Sie die Menge $A\setminus B$ an.
\item Was ist $\mathscr{P}(C)$?
\item Geben Sie eine bijektive Abbildung $f:A\to\mathscr{P}(C)$ an.

\end{enumerate}

\begin{loesung}
\begin{enumerate}
\item $A\cup B=\{1,2,3,\{4\},4\}$
\item $A\cap B=\{2\}$
\item $A\setminus B=\{1,3,\{4\}\}$
\item $\mathscr{P}(C)=\{\emptyset,\{\emptyset\},\{\{1\}\},\{\emptyset,\{1\}\}\}$
\item $f(1)=\emptyset, f(2)=\{\emptyset\}, f(3)=\{\{1\}\}, f(\{4\})=\{\emptyset,\{1\}\}$
\end{enumerate}
\end{loesung}

\aufgabetitel{$12$}{Relationen}\\
Gegeben seien die Relationen $R,S\subseteq\{1,2,3,4,5,6\}$ mit $R=\{(1,2),(2,3),(4,3),(5,4),(6,4)\}$\\und $S=\{(1,1),(1,2),(2,1)\}$.
\begin{enumerate}[(i)]
    \item Stellen Sie $R$ und $S$ als Graphen und Adjazenzmatrix dar.
    \item Stellen Sie die Relation $R\circ S$ als Graphen dar.
    \item Stellen Sie die Relation $S\circ R$ als Graphen dar.
    \item $R$ ist nicht transitiv. Begründen Sie warum (Gegenbeispiel).
    \item Ergänzen Sie den Graphen von $R$ zur reflexiv-transitiven Hülle $R_{trans}$ von $R$ (also nochmal neu zeichnen als reflexiv-transitive Hülle).
    \item $R_{trans}$ ist eine Ordnungsrelation. Was sind die minimalen, maximalen, kleinsten, gr"o{\ss}ten Elemente (sofern existent) davon? Begründen Sie Ihre Antwort.

\end{enumerate}

\begin{loesung}
Aufzeichnungen s. Vorlesung (straight-forward).
\end{loesung}


\aufgabetitel{$6$}{"Aquivalenzrelationen und Partitionen}\\
Sei $M=\{1,2,3,4,5\}$, $A=\{1,2,3\}$, also $A\subseteq M$ und die Partition $P$ auf $M$ mit $P=\{A, \{4\}, \{5\}\}$ gegeben.
\begin{enumerate}[(i)]
    \item Warum ist $P$ eine Partition auf $M$? 
    \item Eine Partition induziert eine "Aquivalenzrelation. Definieren Sie die zu $P$ passend "Aquivalenzrelation.
    \item Argumentieren Sie warum Ihre Relation eine "Aquivalenzrelation ist.
    \item Geben Sie die "Aquivalenzklasse $[1]$ an.
\end{enumerate}
\begin{loesung}
\begin{enumerate}

\item Eine Partition $P$ von $M$ ist eine Menge von Teilmengen von $M$, also eine Teilmenge der Potenzmenge von $M$, so dass je paarweise verschiedene Mengen aus $P$ disjunkt sind (Schnitt leer ist) und die Vereinigung aller Mengen aus $P$ die gesamte Menge $M$ ergibt. Das ist hier der Fall.
\item $R\subseteq M\times M$ mit $(x,y)\in R \Leftrightarrow x\in A \wedge y\in A \vee x=y$.
\item reflexiv: wegen $x=y$ im zweiten Teil der Bedingung, symmetrisch: klar, entweder beide in $A$ oder beide gleich, transitiv: entweder alle in $A$ oder alle gleich.
\item $[1]=A$
\end{enumerate} 
\end{loesung}


\aufgabetitel{$4$}{Indexmengen und Beweis}\\
Es sei $A_i=\{n\in\mathbb{N}\mid n<i\}$.
\begin{enumerate}[(i)]
    \item Bestimmen Sie $A_4$.
    \item Zeigen Sie, dass $\bigcap_{i\in\mathbb{N}}A_i=\emptyset$.
\end{enumerate}

\begin{loesung}

$A_4=\{0,1,2,3\}$\\
\vspace{0.3cm}\\
Beweis:\\
$\supseteq$ ist trivial ($\emptyset$ ist Teilmenge jeder Menge).\\
$\subseteq:$ Es ist $x\in\bigcap_{i\in\mathbb{N}}A_i \Leftrightarrow \forall i\in\mathbb{N}: x\in A_i \Leftrightarrow \forall i\in\mathbb{N}: x<i$. Da es aber keine natürliche Zahl gibt, die kleiner als alle natürlichen Zahlen ist, ist die Schnittmenge leer.
\qed
\end{loesung}

\aufgabetitel{$5$}{Funktionen und Relationen}\\
Sei $f:X\to Y$ eine Funktion und $\sim$ eine Relation "uber $X$ mit $x\sim y:\Leftrightarrow f(x)\leq f(y)$.
\begin{enumerate}[(i)]
    \item Stellen Sie die Funktion $f$ und die Relation $\sim$ f"ur $X=\{1,2,3,4,5\}$, $Y=\{1,2,3\}$, $1\mapsto 1, 2\mapsto 1, 3\mapsto 2, 4\mapsto 2, 5\mapsto 3$ dar.
    \item Zeigen Sie: $\sim$ ist eine Ordnungsrelation genau dann, wenn $f$ injektiv ist.
\end{enumerate}

\begin{loesung}
$\sim\subseteq X\times X$ mit $)1\sim 1, 1\sim 2, 1\sim 3, 1\sim 4, 1\sim 5, 2\sim 2, 2\sim 3, 2\sim 4, 2\sim 5, 3\sim 3, 3\sim 4, 3\sim 5, 4\sim 4, 4\sim 5, 5\sim 5$\\
Beweis:\\
$\Rightarrow:$ Sei $\sim$ eine Ordnungsrelation. Wir zeigen $f$ ist injektiv. Sei dazu $x,y\in X$ und $f(x)=f(y)$. Wir wollen zeigen $x=y$. Da $f(x)=f(y)$ ist insbesondere $f(x)\leq f(y)$ und $f(y)\leq f(x)$, also $x\sim y$ und $y\sim x$. Da $\sim$ antisymmetrisch ist, folgt $x=y$.\\
$\Leftarrow:$ Sei nun $f$ injektiv. Wir wollen zeigen, dass $\sim$ eine Ordnungsrelation ist. Dazu zeigen wir die Reflexivität, die Antisymmetrie und die Transitivität. Reflexivität: Wir müssen zeigen $x\sim x$. Das gilt aber, da $f(x)=f(x)$ für beliebiges $x\in X$ und damit $f(x)\leq f(x)$, also $x\sim x$. Antisymmetrie: Falls $x\sim y$ und $y\sim x$, dann ist $f(x)\leq f(y)$ und $f(y)\leq f(x)$. Wegen der Antisymmetrie von $\leq$ folgt damit $f(x)=f(y)$ und wegen der Injektivität von $f$ folgt damit $x=y$. Transitivität: Falls $x\sim y$ und $y\sim z$, dann ist $f(x)\leq f(y)$ und $f(y)\leq f(z)$. Wegen der Transitivität von $\leq$ folgt damit $f(x)\leq f(z)$ und damit $x\sim z$.\qed
\end{loesung}
\hspace{5cm}
\newpage
\underline{\textbf{Formelsammlung}}\\
Hier eine kleine Formelsammlung. Sie ist nicht vollständig, enthält aber alle wichtigen Statements / Definitionen, die man brauchen könnte.
\begin{enumerate}
\item Aussagen- und Pr"adikatenlogik
\begin{enumerate}
    \item Distributivgesetz: $A\wedge (B \vee C) \Leftrightarrow (A\wedge B) \vee (A\wedge C)$
    \item Distributivgesetz: $A\vee (B \wedge C) \Leftrightarrow (A\vee B) \wedge (A\vee C)$
    \item DeMorgan: $\neg(A\wedge B) \Leftrightarrow \neg A \vee \neg B$
    \item DeMorgan: $\neg(A\vee B) \Leftrightarrow \neg A \wedge \neg B$
    \item Idempotenz: $A\wedge A \Leftrightarrow A$
    \item Idempotenz: $A\vee A \Leftrightarrow A$
    \item $A\wedge \neg A \Leftrightarrow \bot$
    \item $A\vee \neg A \Leftrightarrow \top$
    \item $\neg\neg A \Leftrightarrow A$
    \item $\neg\forall x\in M: A(x) \Leftrightarrow \exists x\in M: \neg A(x)$
    \item $\neg\exists x\in M: A(x) \Leftrightarrow \forall x\in M: \neg A(x)$
\end{enumerate}
\item Mengen
\begin{enumerate}
    \item Teilmenge: $A\subseteq B \Leftrightarrow \forall x\in A: x\in B$
    \item Potenzmenge: $\mathscr{P}(A) = \{B\mid B\subseteq A\}$
    \item Vereinigung: $A\cup B = \{x\mid x\in A \vee x\in B\}$
    \item Schnittmenge: $A\cap B = \{x\mid x\in A \wedge x\in B\}$
    \item Differenzmenge: $A\setminus B = \{x\mid x\in A \wedge x\notin B\}$
    \item Distributivgesetz: $A\cap (B \cup C) \Leftrightarrow (A\cap B) \cup (A\cap C)$
    \item Distributivgesetz: $A\cup (B \cap C) \Leftrightarrow (A\cup B) \cap (A\cup C)$
    \item DeMorgan: $A\setminus (B\cup C) \Leftrightarrow (A\setminus B) \cap (A\setminus C)$
    \item DeMorgan: $A\setminus (B\cap C) \Leftrightarrow (A\setminus B) \cup (A\setminus C)$
    \item Es ist $\bigcup_{i\in I}A_i = \{x\mid \exists i\in I: x\in A_i\}$. 
    \item Es ist $\bigcap_{i\in I}A_i = \{x\mid\forall i\in I: x\in A_i\}$.
\end{enumerate}
\item Relationen
\begin{enumerate}
    \item F"ur Mengen $M,N$ ist $R\subseteq M\times N$ eine Relation von $M$ nach $N$.
    \item $R\subseteq M\times M$ ist reflexiv, wenn $\forall x\in M: (x,x)\in R$.
    \item $R\subseteq M\times M$ ist symmetrisch, wenn $\forall x,y\in M: (x,y)\in R \Rightarrow (y,x)\in R$.
    \item $R\subseteq M\times M$ ist antisymmetrisch, wenn $\forall x,y\in M: (x,y)\in R \wedge (y,x)\in R \Rightarrow x=y$.
    \item $R\subseteq M\times M$ ist transitiv, wenn $\forall x,y,z\in M: (x,y)\in R \wedge (y,z)\in R \Rightarrow (x,z)\in R$.
    \item $R\subseteq M\times M$ ist eine "Aquivalenzrelation, wenn $R$ reflexiv, symmetrisch und transitiv ist.
    \item $R\subseteq M\times M$ ist eine Ordnungsrelation, wenn $R$ reflexiv, antisymmetrisch und transitiv ist.
    \item F"ur eine "Aquivalenzrelation $\sim$ auf $M$ ist $[x]=\{y\in M\mid x\sim y\}$ die "Aquivalenzklasse von $x$, $M/\sim = \{[x]\mid x\in M\}$ die Menge der "Aquivalenzklassen oder Quotientenmenge von $M$ modulo $\sim$. Die Menge der "Aquivalenzklassen ist eine Partition von $M$. Umgekehrt induziert jede Partition eine "Aquivalenzrelation.
    \item F"ur eine Ordnungsrelation $\leq$ auf $M$ und $X\subseteq M$ ist $g$ ein kleinstes Element von $X$, wenn $\forall x\in X: g\leq x$, $g$ ein minimales Element von $X$, wenn $\forall g'\in X: g'\leq g\Rightarrow g=g'$, maximale und gr"o{\ss}te Elemente analog.
\end{enumerate}
\item Funktionen
\begin{enumerate}
    \item Eine Funktion $f:X\to Y$ ist eine Relation (also $f\subseteq X\times Y$), die jedem Element aus der Definitionsmenge $X$ genau ein Element aus der Zielmenge $Y$ zuordnet.
    \item $f$ ist injektiv, wenn $\forall x_1,x_2\in X: f(x_1)=f(x_2) \Rightarrow x_1=x_2$.
    \item $f$ ist surjektiv, wenn $\forall y\in Y: \exists x\in X: f(x)=y$.
    \item $f$ ist bijektiv, wenn $f$ injektiv und surjektiv ist.
    \item Die Umkehrfunktion $f^{-1}$ ist definiert $f^{-1}(Y)=\{x\in X\mid\exists y\in Y: y=f(x)\}$
\end{enumerate}
\end{enumerate}
\end{document}