%!TEX TS-program = pdflatex
%!TEX TS-options = -shell-escape
% % % % %   Die folgenden Zeilen müssen ihre Zeilennummern 4 und 5 behalten !!!    % % % % %
\newcommand{\printpraesenzlsg}{false}
\newcommand{\printloesungen}{true}
\newcommand{\printbewertungen}{false}
% % % % %   \newcommand{\printloesungen}{false}                                    % % % % %
\newcommand{\blattnummer}{1}
%\newcommand{\abgabetermin}{\textcolor{red}{bis 11.04.2022, 09:00 Uhr}}
\input{include/config.tex}

% Änderungen 2020: Hinweise auf Digitallehre angepasst; Rechnergeschichte entfernt; Blatt 1 und 2 zusammengefasst.
\begin{document}

\iforiginal{\thispagestyle{scrplain}
\vspace*{-3cm}
\begin{minipage}[t][1.1cm][c]{4.5cm}
  \includegraphics[width=4cm]{include/hwr-logo.png}
\end{minipage}
\hfill
\begin{minipage}[t][1.5cm][c]{8cm}
  \begin{center}
  \begin{footnotesize}
    \textsf{Prof. Dr.-Ing. Sebastian Schlesinger} \\[-0.1cm]
    \textsf{Fachbereich 2 - Duales Studium Wirtschaft \& Technik}
    %\url{https://www.hwr-berlin.de}
  \end{footnotesize}
  \end{center}
\end{minipage}
\hfill
%\begin{minipage}[t][1.6cm][c]{3.2cm}
%  \includegraphics[width=2.8cm]{include/ESlogo2}
%\end{minipage}

\vspace*{-0.3cm}
\begin{center} 
  \hrulefill \\[0.1cm]
  {\large Übungsblatt } \blattnummer \\[0.15cm]
  {\huge \bfseries Mathematik I - Theoretische Informatik} \\[0.10cm]
  {HWR Berlin, Wintersemester 2025} \\[-0.4cm]
  \begin{tabular}{lcr}
    \hspace{0.3\textwidth}   & \hspace{0.3\textwidth} & \hspace{0.3\textwidth} \\
    Prof. Dr.-Ing. Sebastian Schlesinger   %&                        & "Ubung zur Klausurvorbereitung    \\ 
    %Blatt \blattnummer     % \abgabetermin          \\ 
  \end{tabular} \\[0.1cm]
  \hrulefill
\end{center}}
\aufgabetitel{$5$}{Mengen und Funktionen}\\
Gegeben seien die Mengen $A=\{1,2,3,4\}$, $B=\{2,4,6,8\}$ und $C=\{1,2\}$.
\begin{enumerate}
\item Geben Sie die Menge $A\cup B$ an
\item Geben Sie die Menge $A\cap B$ an
\item Geben Sie die Menge $A\setminus B$ an.
\item Was ist $\mathscr{P}(C)$?
\item Geben Sie eine bijektive Abbildung $f:A\to\mathscr{P}(C)$ an.

\end{enumerate}
\begin{loesung}
\begin{enumerate}
\item $A\cup B=\{1,2,3,4,6,8\}$
\item $A\cap B=\{2,4\}$
\item $A\setminus B=\{1,3\}$
\item $\mathscr{P}(C)=\{\emptyset,\{1\},\{2\},\{1,2\}\}$
\item $f(1)=\{1\}, f(2)=\{2\}, f(3)=\emptyset, f(4)=\{1,2\}$
\end{enumerate}
\end{loesung}

\aufgabetitel{$10$}{Relationen}\\
Gegeben seien die Relationen $R,S\subseteq\{1,2,3,4\}$ mit $R=\{(1,2),(3,1),(4,1)\}$ und $S=\{(1,2),(2,1),(1,3),(4,3)\}$.
\begin{enumerate}[(i)]
    \item Stellen Sie $R$ und $S$ als Graphen und Adjazenzmatrix dar.
    \item Geben Sie die Relation $R\circ S$ an und stellen Sie sie auch als Graphen dar.
    \item Geben Sie die Relation $S\circ R$ an und stellen Sie sie auch als Graphen dar.
    \item Ist $R$ eine Äquivalenzrelation oder Ordnungsrelation? Begründen Sie Ihre Antwort.
    \item Berechnen Sie die reflexiv-transitive Hülle von $R$.
    \item Ist die reflexiv-transitive Hülle von $R$ eine Äquivalenzrelation oder Ordnungsrelation? Begründen Sie Ihre Antwort.

\end{enumerate}
\begin{loesung}
    Ich überlasse das als Übung (straight-forward die Graphen zeichnen und die Matrizen berechnen).
\end{loesung}

\aufgabetitel{$5$}{"Aquivalenzrelation}\\
Sei $x\sim y\Leftrightarrow x=y \vee x=-y$ eine Relation auf $\mathbb{Z}\times\mathbb{Z}$. 
\begin{enumerate}[(i)]
    \item Zeigen Sie, dass $\sim$ eine Äquivalenzrelation ist.
    \item Bestimmen Sie die Quotientenmenge $\mathbb{Z}/\sim$.
\end{enumerate}
\begin{loesung}
Beweis:\\
Reflexivität: Wir müssen zeigen, dass $x\sim x$. Das ist klar, da $x=x$.\\
Symmetrie: Wir müssen zeigen, dass $x\sim y\Rightarrow y\sim x$. Das ist klar, da $x=y\Rightarrow y=x$ und $x=-y\Rightarrow y=-x$.\\
Transitivität: Wir müssen zeigen, dass $x\sim y\wedge y\sim z\Rightarrow x\sim z$. Das zeigt man durch Fallunterscheidung.\qed\\
Es ist $\mathbb{Z}/\sim=\{\{x,-x\}|x\in\mathbb{Z}\}$.
\end{loesung}

\aufgabetitel{$7$}{Ordnungen}\\
Sei $\sim$ eine Relation auf $(\mathbb{N}\times\mathbb{N})\times(\mathbb{N}\times\mathbb{N})$. 
%Wir definieren $((a,b),(c,d))\sim((e,f),(g,h))\Leftrightarrow a+b=c+d \wedge e+f=g+h$. Zeigen Sie, dass $\sim$ eine Äquivalenzrelation ist und bestimmen Sie die Quotientenmenge $(\mathbb{N}\times\mathbb{N})\times(\mathbb{N}\times\mathbb{N})/\sim$.
Wir definieren $(a,b)\sim(c,d)\Leftrightarrow a\leq c \wedge b\leq d$. 
\begin{enumerate}[(i)]
\item Zeigen Sie, dass $\sim$ eine Ordnungsrelation ist.
\item Zeichnen Sie das Hasse-Diagramm f"ur die Teilmenge $\{0,1,2\}\times\{0,1,2\}$.
\item Was sind die gr"o"sten und kleinsten Elemente, maximalen und minimalen Elemente von $\{0,1,2\}\times\{0,1,2\}\backslash\{(2,2)\}$?
\end{enumerate}
\begin{loesung}
Beweis:\\
Reflexivität: Wir müssen zeigen, dass $x\sim x$. Das ist klar, da $a\leq a$ und $b\leq b$.\\
Antisymmetrie: Wir müssen zeigen, dass $x\sim y\wedge y\sim x\Rightarrow x=y$. Das ist klar, da $a\leq c\wedge b\leq d\wedge c\leq a\wedge d\leq b\Rightarrow a=c\wedge b=d$.\\
Transitivität: Wir müssen zeigen, dass $x\sim y\wedge y\sim z\Rightarrow x\sim z$. Das ist klar, da $a\leq c\wedge b\leq d\wedge c\leq e\wedge d\leq f\Rightarrow a\leq e\wedge b\leq f$.\qed\\
Hasse-Diagramm überlass ich als Übung. Das kleinste Element und das einzige minimale Element ist $(0,0)$, es gibt kein größtes Element. Die maximalen Elemente sind $(2,0)$ und $(0,2)$.
\end{loesung}

\aufgabetitel{$6$}{Beweise}\\
Zeigen Sie
\begin{enumerate}[(i)]
    \item $B\backslash (B\backslash A)=A\cap B$ f"ur Mengen $A,B$.
    \item $f^{-1}(B_1\cap B_2)=f^{-1}(B_1)\cap f^{-1}(B_2)$ f"ur Abbildungen $f:A\to B$ und $B_1,B_2\subseteq B$. 
    
    (Hinweis: Es ist $f^{-1}(X)=\{a\in A\mid f(a)\in X\}$.)
\end{enumerate}
\begin{loesung}
Erster Beweis:\\
Sei $x\in B\backslash (B\backslash A)$.\\
Dies ist äquivalent zu $x\in B\wedge x\notin (B\backslash A)$.\\
Dies ist äquivalent zu $x\in B\wedge \neg (x\in B\wedge x\notin A)$.\\
Dies ist äquivalent zu $x\in B\wedge (x\notin B\vee x\in A)$. (De Morgan)\\
Dies ist äquivalent zu $(x\in B\wedge x\notin B)\vee (x\in B\wedge x\in A)$. (Distributivgesetz)\\
Dies ist äquivalent zu $x\in B\wedge x\in A$, was wiederum zu $x\in A\cap B$ äquivalent ist.\qed\\
Zweiter Beweis:\\
Sei $x\in f^{-1}(B_1\cap B_2)$. Das bedeutet, dass $f(x)\in B_1\cap B_2$, also $x\in f^{-1}(B_1)\wedge x\in f^{-1}(B_2)$, also $x\in f^{-1}(B_1)\cap f^{-1}(B_2)$.\qed
\end{loesung}
\end{document}