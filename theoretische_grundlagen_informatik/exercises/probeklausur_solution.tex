%!TEX TS-program = pdflatex
%!TEX TS-options = -shell-escape
% % % % %   Die folgenden Zeilen müssen ihre Zeilennummern 4 und 5 behalten !!!    % % % % %
\newcommand{\printpraesenzlsg}{false}
\newcommand{\printloesungen}{true}
\newcommand{\printbewertungen}{false}
% % % % %   \newcommand{\printloesungen}{false}                                    % % % % %
\newcommand{\blattnummer}{1}
%\newcommand{\abgabetermin}{\textcolor{red}{bis 11.04.2022, 09:00 Uhr}}
\input{include/config.tex}

% Änderungen 2020: Hinweise auf Digitallehre angepasst; Rechnergeschichte entfernt; Blatt 1 und 2 zusammengefasst.
\begin{document}

\iforiginal{\thispagestyle{scrplain}
\vspace*{-3cm}
\begin{minipage}[t][1.1cm][c]{4.5cm}
  \includegraphics[width=4cm]{include/hwr-logo.png}
\end{minipage}
\hfill
\begin{minipage}[t][1.5cm][c]{8cm}
  \begin{center}
  \begin{footnotesize}
    \textsf{Prof. Dr.-Ing. Sebastian Schlesinger} \\[-0.1cm]
    \textsf{Fachbereich 2 - Duales Studium Wirtschaft \& Technik}
    %\url{https://www.hwr-berlin.de}
  \end{footnotesize}
  \end{center}
\end{minipage}
\hfill
%\begin{minipage}[t][1.6cm][c]{3.2cm}
%  \includegraphics[width=2.8cm]{include/ESlogo2}
%\end{minipage}

\vspace*{-0.3cm}
\begin{center} 
  \hrulefill \\[0.1cm]
  {\large Übungsblatt } \blattnummer \\[0.15cm]
  {\huge \bfseries Mathematik I - Theoretische Informatik} \\[0.10cm]
  {HWR Berlin, Wintersemester 2025} \\[-0.4cm]
  \begin{tabular}{lcr}
    \hspace{0.3\textwidth}   & \hspace{0.3\textwidth} & \hspace{0.3\textwidth} \\
    Prof. Dr.-Ing. Sebastian Schlesinger   %&                        & "Ubung zur Klausurvorbereitung    \\ 
    %Blatt \blattnummer     % \abgabetermin          \\ 
  \end{tabular} \\[0.1cm]
  \hrulefill
\end{center}}
\aufgabetitel{$4$}{Mengenoperationen}\\
Gegeben seien die Mengen $A=\{a,b,3,4\}$ und $B=\{1,2,a,b\}$.
\begin{enumerate}[(i)]
    \item Bestimmen Sie $A\cap B$.
    \item Bestimmen Sie $A\cup B$.
    \item Bestimmen Sie $A\backslash B$.
    \item Bestimmen Sie $\mathscr{P}(A\cap B)$
    
\end{enumerate}

\begin{loesung}
\begin{enumerate}
\item $A\cap B=\{a,b\}$
\item $A\cup B=\{1,2,a,b,3,4\}$
\item $A\backslash B=\{3,4\}$
\item $\mathscr{P}(A\cap B)=\{\emptyset,\{a\},\{b\},\{a,b\}\}$
\end{enumerate}
\end{loesung}

\aufgabetitel{$2$}{"Uberlegungen mit Mengen}\\
Die beiden folgenden Aussagen gelten nicht allgemein f"ur alle Mengen $A,B,C$. Finden Sie Gegenbeispiele f"ur Mengen $A,B,C$ zu folgenden Aussagen:
\begin{enumerate}[(i)]
\item $A\subseteq B\Rightarrow A\cap B\neq\emptyset$
\item $A\cap B\neq\emptyset\wedge B\cap C\neq\emptyset\Rightarrow A\cap C\neq\emptyset$
\end{enumerate}

\begin{loesung}
\begin{enumerate}
\item $A=\emptyset, B=\{1\}$
\item $A=\{1,2\},B=\{1\}, C=\{2\}$
\end{enumerate}
\end{loesung}

\aufgabetitel{$2$}{Mengenbeweis}\\
Seien $M,N$ Mengen. Zeigen Sie, dass folgendes gilt:
\[\mathscr{P}(M)\cap\mathscr{P}(N)\subseteq \mathscr{P}(M\cap N)\]

\begin{loesung}
Beweis:\\
Sei $X\in\mathscr{P}(M)\cap\mathscr{P}(N)$. Dann ist $X\in\mathscr{P}(M)$ und $X\in\mathscr{P}(N)$. Also ist $X\subseteq M$ und $X\subseteq N$. Daraus folgt $X\subseteq M\cap N$. Also ist $X\in\mathscr{P}(M\cap N)$.\qed
\end{loesung}

\aufgabetitel{$14$}{Relationen}\\
Gegeben seien die Relation $R\subseteq M\times M$ und $S\subseteq M\times M$ mit $M=\{1,2,3,4\}$, 

$R=\{(1,1),(1,2),(1,4),(2,1),(2,3),(3,2),(4,3)\}$ und $S=\{(1,2),(2,2),(1,4)\}$.
\begin{enumerate}[(i)]
\item Stellen Sie $R$ und $S$ als Adjazenzmatrix und Graph dar.
\item Sind $R$ oder $S$ Ordnungen? Begr"unden Sie Ihre Entscheidung.\\
\textit{(Hinweis zur Erinnerung: Eine Ordnung ist eine Relation, die reflexiv, antisymmetrisch und transitiv ist. Sie muss nicht unbedingt total sein (Total w"urde bedeuten, dass jeweils zwei Elemente immer in Relation stehen - in der einen oder anderen Richtung, also $\forall x,y\in M:(x,y)\in M\vee (y.x)\in M$, was wir in der Vorlesung und auch hier nicht fordern.)).}
\item Berechnen Sie $R\circ S$ und stellen Sie das Ergebnis als Adjazenzmatrix und Graph dar.\\
\textit{(Hinweis zur Erinnerung: Es ist $R\circ S=\{(x,y)\in M\times M|\exists z\in M: (x,z)\in R\wedge (z,y)\in S\}$)}
\item Wir bezeichnen im Folgenden $T=R\circ S$ ($T$ als Abk"urzung von $R\circ S$). Berechnen Sie $T\circ T$ und stellen Sie diese Relation als Adjazenzmatrix und Graph dar.
\item Sei $Id=\{(1,1),(2,2),(3,3),(4,4)\}$ die Relation, die jedes Element von $M$ mit sich selbst in Beziehung setzt. Geben Sie die Adjazenzmatrix und den Graph der Relation $O=Id\cup T\cup T\circ T$ an.
\item Begr"unden Sie, dass $O$ eine Ordnung ist.
\item Stellen Sie $O$ als Hasse-Diagramm dar.

\end{enumerate}

\begin{loesung}
Das belass ich als Übung (einfach die Graphen zeichnen und die Matrizen ausrechnen).
\end{loesung}

\aufgabetitel{$5$}{Relationen und Eigenschaften}\\
Entscheiden Sie f"ur jede Aussage, ob sie zutrifft, begr"unden Sie Ihre Entscheidung und geben Sie ggf. ein Beispiel an.

Es gibt eine Relation $R\subseteq M\times M$ "uber einer Menge $M$, die 
\begin{enumerate}[(i)]
\item reflexiv und irreflexiv ist\\
\textit{(Hinweis zur Erinnerung: $R$ ist reflexiv, wenn $\forall x\in M:(x,x)\in R$, irreflexiv, wenn $\forall x\in M:(x,x)\notin R$)}
\item weder reflexiv noch irreflexiv ist
\item symmetrisch und antisymmetrisch ist\\
\textit{(Hinweis zur Erinnerung: $R$ ist symmetrisch, wenn $\forall x,y\in M:(x,y)\in R\Rightarrow (y,x)\in R$, antisymmetrisch, wenn $\forall x,y\in M:(x,y)\in R\wedge (y,x)\in R\Rightarrow x=y$)}
\item antisymmetrisch und irreflexiv ist
\item symmetrisch und antisymmetrisch ist.
\end{enumerate}

\begin{loesung}
\begin{enumerate}[(i)]
\item geht nicht, weil sich die Bedingungen ausschlie"sen.
\item $R=\{(1,1)\}$ auf $M=\{1,2\}$
\item $R=\emptyset$
\item $R=\emptyset$
\item $R=\emptyset$
\end{enumerate}
\end{loesung}
\aufgabetitel{$4$}{Beweis mit Relationen}\\
Seien $R_1,S_1,R_2,S_2\subseteq M\times M$ Relationen auf einer Menge $M$. Zeigen Sie \[R_1\subseteq R_2\wedge S_1\subseteq S_2\Rightarrow R_1\circ S_1\subseteq R_2\circ S_2\]

\begin{loesung}
Beweis:\\
Sei $(x,y)\in R_1\circ S_1$. Dann existiert ein $z\in M$ mit $(x,z)\in R_1$ und $(z,y)\in S_1$. Da $R_1\subseteq R_2$ und $S_1\subseteq S_2$ folgt $(x,z)\in R_2$ und $(z,y)\in S_2$. Also ist $(x,y)\in R_2\circ S_2$.\qed
\end{loesung}
\aufgabetitel{$2$}{Funktionen}\\
Geben Sie f"ur die folgenden Funktionen an, ob sie injektiv oder surjektiv sind und begr"unden Sie Ihre Entscheidung.\\
\textit{(Hinweis zur Erinnerung:\\$f:M\to N$ ist injektiv, wenn $\forall x,y\in M:f(x)=f(y)\Rightarrow x=y$ und surjektiv, wenn $\forall y\in N\exists x\in M:y=f(x)$)}

\begin{enumerate}[(i)]
\item $f:\RR\to\RR, x\mapsto x$
\item $f:\ZZ\to\NN, x\mapsto 1$
\end{enumerate}

\end{document}