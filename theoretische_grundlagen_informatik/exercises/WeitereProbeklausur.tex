%!TEX TS-program = pdflatex
%!TEX TS-options = -shell-escape
% % % % %   Die folgenden Zeilen müssen ihre Zeilennummern 4 und 5 behalten !!!    % % % % %
\newcommand{\printpraesenzlsg}{false}
\newcommand{\printloesungen}{false}
\newcommand{\printbewertungen}{false}
% % % % %   \newcommand{\printloesungen}{false}                                    % % % % %
\newcommand{\blattnummer}{1}
%\newcommand{\abgabetermin}{\textcolor{red}{bis 11.04.2022, 09:00 Uhr}}
\input{include/config.tex}

% Änderungen 2020: Hinweise auf Digitallehre angepasst; Rechnergeschichte entfernt; Blatt 1 und 2 zusammengefasst.
\begin{document}

\iforiginal{\thispagestyle{scrplain}
\vspace*{-3cm}
\begin{minipage}[t][1.1cm][c]{4.5cm}
  \includegraphics[width=4cm]{include/hwr-logo.png}
\end{minipage}
\hfill
\begin{minipage}[t][1.5cm][c]{8cm}
  \begin{center}
  \begin{footnotesize}
    \textsf{Prof. Dr.-Ing. Sebastian Schlesinger} \\[-0.1cm]
    \textsf{Fachbereich 2 - Duales Studium Wirtschaft \& Technik}
    %\url{https://www.hwr-berlin.de}
  \end{footnotesize}
  \end{center}
\end{minipage}
\hfill
%\begin{minipage}[t][1.6cm][c]{3.2cm}
%  \includegraphics[width=2.8cm]{include/ESlogo2}
%\end{minipage}

\vspace*{-0.3cm}
\begin{center} 
  \hrulefill \\[0.1cm]
  {\large Übungsblatt } \blattnummer \\[0.15cm]
  {\huge \bfseries Mathematik I - Theoretische Informatik} \\[0.10cm]
  {HWR Berlin, Wintersemester 2025} \\[-0.4cm]
  \begin{tabular}{lcr}
    \hspace{0.3\textwidth}   & \hspace{0.3\textwidth} & \hspace{0.3\textwidth} \\
    Prof. Dr.-Ing. Sebastian Schlesinger   %&                        & "Ubung zur Klausurvorbereitung    \\ 
    %Blatt \blattnummer     % \abgabetermin          \\ 
  \end{tabular} \\[0.1cm]
  \hrulefill
\end{center}}
\aufgabetitel{$4$}{Mengenoperationen}\\
Gegeben seien die Mengen $A=\{(1,1),(2,2),(3,2)\}$ und $B=\{(1,1),2,a,b\}$.
\begin{enumerate}[(i)]
    \item Bestimmen Sie $A\cap B$.
    \item Bestimmen Sie $A\cup B$.
    \item Bestimmen Sie $A\backslash B$.
    \item Bestimmen Sie $\mathscr{P}(A\cap B)$
    
\end{enumerate}

\aufgabetitel{$2$}{Intervall als Mengendeklaration}\\
Ein Intervall auf einer Grundmenge, z.B. $\mathbb{R}$ ist eine Menge von Zahlen, die zwischen zwei Grenzen liegen.
Man schreibt dann z.B. $[a,b]$ für die Mengen der Zahlen, die zwischen $a$ und $b$ liegen, wobei $a$ und $b$ mit dazugehören.\\
Definieren Sie das Intervall $[a,b]\subseteq\mathbb{R}$ als Mengendeklaration.

\aufgabetitel{$10$}{Relationen und Funktionen}\\ 
Sei $X=\{1,2,3,4,5,6,7,8\}$, $Y=\{a,b,c\}$ und $f\subseteq X\times Y$ mit 

$f=\{(1,a),(2,a),(3,c),(4,b),(5,a),(6,b),(7,c),(8,a)\}$.
\begin{enumerate}[(i)]
    \item Ist $f$ eine Funktion? Begründen Sie Ihre Antwort.
    \item Wir definieren eine neue Relation $R\subseteq X\times X$ mit $R=\{(x,y)\in X\times X\mid f(x)=f(y)\}$. Beweisen Sie, dass es eine Äquivalenzrelation ist.
    \item Die Menge der Äquivalenzklassen von $R$ ist definiert als die Menge $X/R=\{[x]\mid x\in X\}$, wobei $[x]=\{y\in X\mid (x,y)\in R\}$ als \textit{Äquivalenzklasse} von $x$ bezeichnet wird (das sind wie definiert alle Elemente, die mit einem gegebenen $x\in M$ in Relation stehen). Bestimmen Sie $X/R$.
\end{enumerate}

\aufgabetitel{$8$}{Relationen}\\
Beantworten Sie die folgenden Fragen.
\begin{enumerate}[(i)]
    \item Was ist eine Relation?
    \item Was ist eine reflexive Relation?
    \item Was ist eine symmetrische Relation?
    \item Was ist eine antisymmetrische Relation?
    \item Was ist eine transitive Relation?
    \item Was ist eine Äquivalenzrelation?
    \item Was ist eine Ordnungsrelation?
    \item Was ist eine Funktion?
\end{enumerate}

\aufgabetitel{$9$}{Relationen}\\
Sei $M=\{1,2,3\}$ und $R\subseteq M\times M$ mit $1R1, 1R2, 2R2, 2R3, 3R1$.
\begin{enumerate}[(i)]
    \item Ist $R$ symmetrisch? Begründen Sie Ihre Antwort.
    \item Ist $R$ antisymmetrisch? Begründen Sie Ihre Antwort.
    \item Ist $R$ transitiv? Begründen Sie Ihre Antwort.
    \item Zeichnen Sie $R$ als Graphen.
    \item Geben Sie $R$ als Adjazaenzmatrix an.
    \item Berechnen Sie $R\circ R$ und geben Sie das Ergebnis als Adjazenzmatrix und Graphen an.
    \item Sei $R^{-1}=\{(y,x)\mid (x,y)\in R\}$ die Relation, die faktisch $R$ umkehrt. Geben Sie $R^{-1}$ als Adjazenzmatrix und Graphen an.
\end{enumerate}
\end{document}